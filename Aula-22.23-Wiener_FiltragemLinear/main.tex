\documentclass{article}
\usepackage{enumitem}
\usepackage[T1]{fontenc}
\usepackage[utf8]{inputenc}
\usepackage[portuguese]{babel}
\usepackage[vmargin=3cm]{geometry}
\usepackage{tikzpagenodes}
\usepackage{lipsum}
\usepackage{xcolor}
\usepackage{cancel}
\usepackage{amsmath}
\usepackage{mathrsfs}
\usepackage{amssymb}
\usepackage{background}
\usepackage{titlesec}
\usepackage[nodisplayskipstretch]{setspace}
\usepackage{hyphenat}
\usepackage[normalem]{ulem}
\usepackage{subcaption}
\hyphenation{mate-mática recu-perar}

\titlespacing{\section}{0pc}{-0.25em}{0pc}
\titlespacing{\subsection}{0pc}{0em}{0pc}
\titlespacing{\subsubsection}{0pc}{0.33em}{0pc}
\titlespacing{\paragraph}{0em}{0.25em}{0.5em}
\setlength{\parindent}{2em}
\setlength{\parskip}{1em}
\linespread{1}

\renewcommand{\baselinestretch}{1.0}

\renewcommand\bf[1]{\textbf{#1}}
\renewcommand\it[1]{\textit{#1}}

\newcommand\ov[1]{\overline{#1}}
\renewcommand\u[1]{\underbar{#1}}
\newcommand{\vect}[1]{\mathbf{#1}}
\newcommand{\bb}[1]{\mathbb{#1}}
\newcommand{\vn}{\varnothing}
\newcommand\stk[2][black]{\setbox0=\hbox{$#2$}%
\rlap{\raisebox{.45\ht0}{\textcolor{#1}{\rule{\wd0}{1pt}}}}#2}

\makeatletter
\global\let\tikz@ensure@dollar@catcode=\relax
\makeatother

\makeatletter
\def\mcolor#1#{\@mcolor{#1}}
\def\@mcolor#1#2#3{%
  \protect\leavevmode
  \begingroup
    \color#1{#2}#3%
  \endgroup
}
\makeatother
\definecolor{notepadrule}{RGB}{217,244,244}

\backgroundsetup{
contents={%
  \begin{tikzpicture}
    \foreach \fila in {0,...,52}
    {
      \draw [line width=1pt,color=notepadrule]
      (current page.west|-0,-\fila*12pt) -- ++(\paperwidth,0);
    }
    \draw[overlay,red!70!black,line width=1pt]
      ([xshift=-1pt]current page text area.west|-current page.north) --
      ([xshift=-1pt]current page text area.west|-current page.south);
  \end{tikzpicture}%
},
scale=1,
angle=0,
opacity=1
}

\begin{document}

\setlength{\abovedisplayskip}{12pt}
\setlength{\belowdisplayskip}{12pt}
\setlength{\abovedisplayshortskip}{0pt}
\setlength{\belowdisplayshortskip}{0pt}
% \setlength{\baselineskip}{12pt}
\setlength{\jot}{1pt}

\section{Teorema de Wiener-Kinchin-Einstein no Caso Discreto}
Processo estocástico $X$, WSS. Com função de autocorrelação (dada pelo intervalo): $r[n]$ e função
de covariância: $c[n]$.

Definição de densidade espectral de potência do processo:
\\
Considere o processo $X_{[-N_1,N_1]}$ dado pelo janelamento de $X$:
\begin{align*}
    X_{[-N_1,N_1]}(s) = X(s) \cdot rect(-N_1,N_1)
\end{align*}
Considere ainda a DTFT do processo:
\begin{align*}
    \widehat{X}_{[-N_1,N_1]}(s) &= DTFT(X_{[-N_1,N_1]}(s)) \\
    PSD_X(f) &= \lim_{N_1 \to \infty} \frac{1}{2N_1+1} \mathbb{E}(|\widehat{X}_{[-N_1,N_1]}^{(f)}|^2)
\end{align*}
Teorema:
\begin{align*}
    \widehat{r}(f) = PSD_X(f) = \lim_{N_1 \to \infty} \frac{1}{2N_1+1} \mathbb{E}(|\widehat{X}_{[-N_1,N_1]}^{(f)}|^2)
\end{align*}

\section{Filtragem de Processos Estocásticos}
Ideia geral: Processo $X[n] \to X^{(n)}$ é entrada de um sistema LTI com resposta impulsional
$h[n]$. \textit{Generalização de uma função de sinais} $g: E_S \to E_S$ definida por $g(x) = h*x$
para processos $g(X(s)) = Y(s) = h * X(s)$.
\\
Considere $R_{XX}$ que é a função de autocorrelação de $X$. $R_{XX}[n_1,n_2] =
\mathbb{E}[X_{n_1}\cdot X_{n_2}^*]$. Queremos a partir de $R_{XX}$ calcular $R_{YY}$:
\begin{align*}
    R_{YY}[n_1,n_2] &= \mathbb{E}[X_{n_1}\cdot X_{n_2}^*] \\
    \text{Considerando: } R_{XY}[n_1,n_2] &= \mathbb{E}[X_{n_1}\cdot Y_{n_2}^*] \text{ (Correlação Cruzada)} \\
    R_{YX}[n_1,n_2] &= \mathbb{E}[Y_{n_1}\cdot X_{n_2}^*] \\
    R_{XY}[n_1,n_2] &= \mathbb{E}[X_{n_1}\cdot Y_{n_2}^*] = \mathbb{E}\left[X_{n_1} \cdot \sum_{m=0}^\infty h^*[m]X_{n_2-m}^*\right] \\
    R_{XY}[n_1,n_2] &=  \sum_{m=0}^\infty h^*[m] \mathbb{E}\left[X_{n_1}\cdot X_{n_2-m}\right] \\
    R_{XY}[n_1,n_2] &=  \sum_{m=0}^\infty h^*[m] R_{XX}[n_1,n_2-m]
\end{align*}

\subsection{Caso WSS}
Se $X$ é estacionário no sentido amplo (WSS), então $R_{XX}[n_1, n_2-m]$ depende apenas de
$n_2-m-n_1$, e não de $n_1$ e $n_2$.

\begin{align*}
    R_{XY}[n_1,n_2] &=  \sum_{m=-\infty}^\infty h^*[m] R_{XX}[n_1,n_2-m] \\
    R_{XY}[n_1,n_2] &=  \sum_{m=-\infty}^\infty h^*[m] r_{XX}[n_2-n_1-m] \\
    r_{XY}[n_1,n_2] &=   (h^* * r_{XX})_{[n_2-n_1-m]}
\end{align*}

$R_{XY}$ também só depende da diferença entre os instantes considerados para o caso WSS.

Buscamos agora a autocorrelação da saída $R_{YY}[n_1,n_2]$:
\begin{align*}
    R_{XY}[n_1,n_2] &= \sum_{m=0}^\infty h^*[m] R_{XX}[n_1,n_2-m] \\
    R_{YY}[n_1,n_2] &= \mathbb{E}[Y_{n_1}\cdot Y_{n_2}ˆ*] \\
    R_{YY}[n_1,n_2] &= \mathbb{E}\left[Y_{n_1}\cdot \sum_{m=-\infty}^\infty h^*[m]X_{n_2-m}^*\right] \\
    R_{YY}[n_1,n_2] &= \sum_{m=-\infty}^\infty h^*[m] \cdot \mathbb{E}\left[Y_{n_1}\cdot X_{n_2-m}^*\right] \\
    R_{YX}[n_1,n_2] &= \mathbb{E}[Y_{n_1}\cdot X_{n_2}^*] = \sum_{m=-\infty}^\infty h[m]\cdot \mathbb{E}[X_{n_1-m} \cdot X_{n_2}^*] \\
    R_{YX}[n_1,n_2] &= \sum_{m=-\infty}^\infty h[m]\cdot R_{XX}[n_1-m,n_2] \\
    R_{YX}[n_1,n_2] &= \sum_{m=-\infty}^\infty h[m]\cdot r_{XX}[-[(n_1-n_2)-m]] \\
    R_{YX}[n_1,n_2] &= \sum_{m=-\infty}^\infty h[m]\cdot \overline{r_{XX}}[(n_1-n_2)-m], \overline{r_{XX}} \text{ é uma reversão no tempo} \\
    R_{YX}[n_1,n_2] &= (h * \overline{r_{XX}})_{[n_1-n_2]} \\
    r_{YX}[n] &= (h * \overline{r_{XX}})_{[n]} \\
    R_{YY}[n_1,n_2] &= \sum_{m=-\infty}^\infty h^*[m] \cdot \mathbb{E}\left[Y_{n_1}\cdot X_{n_2-m}^*\right] \\
    R_{YY}[n_1,n_2] &= \sum_{m=-\infty}^\infty h^*[m] \cdot R_{YX}[n_1,n_2-m]
\end{align*}
Para sinal real (WSS): $r_{YX} = h * r_{XX}$
\\
Note que se o sinal é real e o filtro for real (WSS): $r_{XY} = h^* * r_{XX} = r_{YX} = h * r_{XX}$!

Supondo que $X$ é real e WSS:
\begin{align*}
    r_{YX} &= h * r_{XX} \\
    r_{YY} &= h^* * r_{YX} \\
    r_{YY} &= h^* * (h * r_{XX})
\end{align*}


\end{document}
