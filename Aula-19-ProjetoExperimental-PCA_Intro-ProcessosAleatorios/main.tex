\documentclass{article}
\usepackage{enumitem}
\usepackage[T1]{fontenc}
\usepackage[utf8]{inputenc}
\usepackage[portuguese]{babel}
\usepackage[vmargin=3cm]{geometry}
\usepackage{tikzpagenodes}
\usepackage{lipsum}
\usepackage{xcolor}
\usepackage{cancel}
\usepackage{amsmath}
\usepackage{mathrsfs}
\usepackage{amssymb}
\usepackage{background}
\usepackage{titlesec}
\usepackage[nodisplayskipstretch]{setspace}
\usepackage{hyphenat}
\usepackage[normalem]{ulem}
\usepackage{subcaption}
\hyphenation{mate-mática recu-perar}

\titlespacing{\section}{0pc}{-0.25em}{0pc}
\titlespacing{\subsection}{0pc}{0em}{0pc}
\titlespacing{\subsubsection}{0pc}{0.33em}{0pc}
\titlespacing{\paragraph}{0em}{0.25em}{0.5em}
\setlength{\parindent}{2em}
\setlength{\parskip}{1em}
\linespread{1}

\renewcommand{\baselinestretch}{1.0}

\renewcommand\bf[1]{\textbf{#1}}
\renewcommand\it[1]{\textit{#1}}

\newcommand\ov[1]{\overline{#1}}
\renewcommand\u[1]{\underbar{#1}}
\newcommand{\vect}[1]{\mathbf{#1}}
\newcommand{\bb}[1]{\mathbb{#1}}
\newcommand{\vn}{\varnothing}
\newcommand\stk[2][black]{\setbox0=\hbox{$#2$}%
\rlap{\raisebox{.45\ht0}{\textcolor{#1}{\rule{\wd0}{1pt}}}}#2}

\makeatletter
\global\let\tikz@ensure@dollar@catcode=\relax
\makeatother

\makeatletter
\def\mcolor#1#{\@mcolor{#1}}
\def\@mcolor#1#2#3{%
  \protect\leavevmode
  \begingroup
    \color#1{#2}#3%
  \endgroup
}
\makeatother
\definecolor{notepadrule}{RGB}{217,244,244}

\backgroundsetup{
contents={%
  \begin{tikzpicture}
    \foreach \fila in {0,...,52}
    {
      \draw [line width=1pt,color=notepadrule]
      (current page.west|-0,-\fila*12pt) -- ++(\paperwidth,0);
    }
    \draw[overlay,red!70!black,line width=1pt]
      ([xshift=-1pt]current page text area.west|-current page.north) --
      ([xshift=-1pt]current page text area.west|-current page.south);
  \end{tikzpicture}%
},
scale=1,
angle=0,
opacity=1
}

\begin{document}

\setlength{\abovedisplayskip}{12pt}
\setlength{\belowdisplayskip}{12pt}
\setlength{\abovedisplayshortskip}{0pt}
\setlength{\belowdisplayshortskip}{0pt}
% \setlength{\baselineskip}{12pt}
\setlength{\jot}{1pt}

\section{Experimento com Sinais}
Com sinais de ECG e sEMG posicionamos em cada coluna de uma matriz um pulso do sinal de ECG ou uma
bulha do EMG dinâmico. Na matriz de dados: separar 70\% para cálculo de $T$ e 30\% das realizações
para testes.

O principal objetivo é fazer a análise de componentes principais. Obter a matriz da transformada de
Kahunen-Loève. Alguns resultados esperados:
\begin{itemize}
    \item Observar visualmente a concentração da informação em poucos coeficientes.
    \item Quantificar $l_0$ (qtd. não nulos), $l_1$ (soma dos módulos)
    \item Fazer um experimento de compressão de sinais por transformadas:
    \begin{itemize}
        \item Calcular diferentes transformadas do vetor de teste (DFT, DCT, DST, DWT);
        \item Zerar os $N_b$ com $N_a = N - N_b$ coeficientes de menor magnitude;
        \item Obter a transformada inversa;
        \item Computar a relação sinal-erro (SER);
        \item Plotar a relação SER em função de $N_a$.
    \end{itemize}
\end{itemize}

\section{Processos Estocásticos em Domínio Contínuo e em Domínio Discreto de Duração Infinita}
Considere um experimento aleatório cujos resultados possíveis são elementos do espaço amostral $S$.
Um processo Estocástico $\mathbf{X}$ é uma função que associa resultados do experimento a sinais em
tempo contínuo ou discreto:
\begin{align*}
    \mathbf{X}: S \to E_s,
\end{align*}
com $E_s$ é um espaço de sinais.

Por exemplo, podemos ter $E_s$ como o conjuntos de todos os sinais em tempo contínuo $x_c :
\mathbb{R} \to \mathbb{R}$.
\\
Podemos ter ainda $E_s$ como o conjunto de todos os sinais em tempo discreto $x: \mathbb{Z} \to R$.
\\
Por fim, podemos ter ainda $E_s$ como o conjunto de todos os sinais em tempo discreto com duração
$N$: $x: \{0,1,2,3,\ldots, N\} \to \mathbb{R}$. \textit{É idêntico (isometria) ao caso do vetor
aleatório estabelecido anteriormente.}

\subsection{Observação}
Dado um processo estocástico em tempo contínuo, podemos definir uma função que atua sobre cada
sinal $E_s$, gerando um novo sinal, ou um vetor, ou um único número:

\begin{itemize}
    \item $g: E_s \to E_s^{'}$ - um novo espaço de sinais
    \item $g: E_s \to \mathbb{R}^N$ - espaço de vetores N-dimensionais
    \item $g: E_s \to \mathbb{R}$ - conjunto de números
\end{itemize}
\textit{Essas funções induzem um novo processo aleatório $Y = g(\mathbf{X}_c)$}.

\end{document}
