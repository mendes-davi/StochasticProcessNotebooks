\documentclass{article}
\usepackage{enumitem}
\usepackage[T1]{fontenc}
\usepackage[utf8]{inputenc}
\usepackage[portuguese]{babel}
\usepackage[vmargin=3cm]{geometry}
\usepackage{tikzpagenodes}
\usepackage{lipsum}
\usepackage{xcolor}
\usepackage{cancel}
\usepackage{amsmath}
\usepackage{mathrsfs}
\usepackage{amssymb}
\usepackage{background}
\usepackage{titlesec}
\usepackage[nodisplayskipstretch]{setspace}
\usepackage{hyphenat}
\usepackage[normalem]{ulem}
\usepackage{subcaption}
\hyphenation{mate-mática recu-perar}

\titlespacing{\section}{0pc}{-0.25em}{0pc}
\titlespacing{\subsection}{0pc}{0em}{0pc}
\titlespacing{\subsubsection}{0pc}{0.33em}{0pc}
\titlespacing{\paragraph}{0em}{0.25em}{0.5em}
\setlength{\parindent}{2em}
\setlength{\parskip}{1em}
\linespread{1}

\renewcommand{\baselinestretch}{1.0}

\renewcommand\bf[1]{\textbf{#1}}
\renewcommand\it[1]{\textit{#1}}

\newcommand\ov[1]{\overline{#1}}
\newcommand{\vect}[1]{\mathbf{#1}}
\newcommand{\bb}[1]{\mathbb{#1}}
\newcommand{\vn}{\varnothing}
\newcommand\stk[2][black]{\setbox0=\hbox{$#2$}%
\rlap{\raisebox{.45\ht0}{\textcolor{#1}{\rule{\wd0}{1pt}}}}#2}

\makeatletter
\global\let\tikz@ensure@dollar@catcode=\relax
\makeatother

\makeatletter
\def\mcolor#1#{\@mcolor{#1}}
\def\@mcolor#1#2#3{%
  \protect\leavevmode
  \begingroup
    \color#1{#2}#3%
  \endgroup
}
\makeatother
\definecolor{notepadrule}{RGB}{217,244,244}

\backgroundsetup{
contents={%
  \begin{tikzpicture}
    \foreach \fila in {0,...,52}
    {
      \draw [line width=1pt,color=notepadrule]
      (current page.west|-0,-\fila*12pt) -- ++(\paperwidth,0);
    }
    \draw[overlay,red!70!black,line width=1pt]
      ([xshift=-1pt]current page text area.west|-current page.north) --
      ([xshift=-1pt]current page text area.west|-current page.south);
  \end{tikzpicture}%
},
scale=1,
angle=0,
opacity=1
}

\begin{document}

\setlength{\abovedisplayskip}{12pt}
\setlength{\belowdisplayskip}{12pt}
\setlength{\abovedisplayshortskip}{0pt}
\setlength{\belowdisplayshortskip}{0pt}
% \setlength{\baselineskip}{12pt}
\setlength{\jot}{1pt}

\section{A desigualdade de Chebyshev}
\bf{Problema 4 - Lista 2:} Seja $X$ uma variável aleatória com valor esperado $\mu = E(X)$ e desvio
padrão (suposto não nulo) $\sigma = \sqrt{\text{Var}(X)}$.

A desigualdade de Chebyshev estabelece que
\begin{align*}
P(|X - \mu| \geq k \sigma) \leq \frac{1}{k^2}, \; \forall \; k > 0 \\
\\
| X - \mu | \geq k \sigma =
\begin{cases}
    X \geq \mu + k \sigma \\
    X \leq \mu - k \sigma \\
\end{cases} \\
P(|X - \mu| \geq k \sigma) = P_X((-\infty, \mu-k\sigma] \cup (\mu+k\sigma, \infty]) \\[0.5em]
\end{align*}
\subsection*{A}
$P(|X - \mu| \geq 2k \sigma) = \frac{1}{4}$

\section{Probabilidade \& Frequência Relativa}

Em um experimento de Bernoulli
\begin{align*}
    S = \{(C,C,\ldots,C), \ldots \} \text{ com } |S| = 2^n
\end{align*}
Uma variável aleatória $X(k)$ indica $k$ elementos iguais a $C$ em um resultado possível.  Com
probabilidade $P(\{k\}) = {N \choose k} p^k (1-p)^{N-k}$

Com base no valor esperado de X e na desigualdade de Chebyshev, forneça uma interpretação do conceito de probabilidade em termos de frequência relativa.

Usando a resposta do \bf{Problema 2} para o valor esperado de uma V.A. binomial é $\bb{E}(X) = Np$
e Var$(X) = Np(1-p)$, $\mu = \sqrt{Np(1-p)}$
\begin{align*}
P(|X - \mu| \geq j \sigma) \leq \frac{1}{j^2}, \; \forall \; j > 0 \\
P(|X - Np| \geq j \sqrt{Np(1-p)}) \leq \frac{1}{j^2} \\[-1em]
\end{align*}
Definindo $Y = \frac{X}{N}$ (uma v.a.) como uma medida de frequência relativa para uma proporção de
ocorrências do evento desejado.

Avaliando o valor esperado $\bb{E}(Y) = \frac{Np}{N} = p$ e Var$(Y) = \frac{Var(X)}{N^2} = \frac{p(1-p)}{N}$, $\sigma_Y = \sqrt{\frac{p(1-p)}{N}}$.

Finalmente, aplicando esse valores a desigualdade de Chebyshev.

\begin{align*}
    P(|Y - p| \geq j \sqrt{\frac{p(1-p)}{N}}) \leq \frac{1}{j^2} \\
    tol = j \sqrt{\frac{p(1-p)}{N}} \\
    j^2 = tol^2 \frac{N}{p(1-p)} \\
    \\
    P(|Y - p| \geq tol) \leq \frac{p(1-p)}{N tol^2}
\end{align*}

A probabilidade de que a freq. relativa desvie acima de uma tol. em relação à probabilidade teórica
de acerto tem um limite superior que cai linearmente com o número de realizações N.

\section{Vetores Aleatórios - Introdução}
Vetor que representa várias medidas (eg. amostras de um sinal) de um sistema. \bf{Não definir
vetor aleatório como:} uma lista de variáveis aleatórias.

Definição: Uma função $X: S \rightarrow \mathbb{R}^n$, portanto função que associa um resultado de
um experimento a um vetor numérico. Um vetor aleatório é um tipo de processo estocástico em tempo
discreto e duração finita. [tempo é o índice que define a posição em um vetor de $\mathbb{R}^n$]

Em função disso, podemos definir:
\begin{itemize}
    \item A PDF conjunta de um vetor
    \item A CDF conjunta de um vetor
    \item A dependência ou independência entre amostras do vetor
    \item A covariância entre componentes
    \item A correlação entre componentes
    \item A matriz de covariâncias
    \item A matriz de dependências
\end{itemize}

\end{document}
