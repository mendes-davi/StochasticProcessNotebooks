\documentclass{article}
\usepackage{enumitem}
\usepackage[T1]{fontenc}
\usepackage[utf8]{inputenc}
\usepackage[portuguese]{babel}
\usepackage[vmargin=3cm]{geometry}
\usepackage{tikzpagenodes}
\usepackage{lipsum}
\usepackage{xcolor}
\usepackage{cancel}
\usepackage{amsmath}
\usepackage{amssymb}
\usepackage{background}
\usepackage{titlesec}
\usepackage[nodisplayskipstretch]{setspace}
\usepackage{hyphenat}
\usepackage[normalem]{ulem}
% \usepackage{minted}
\usepackage{subcaption}
\hyphenation{mate-mática recu-perar}

\titlespacing{\section}{0pc}{-0.25em}{0pc}
\titlespacing{\subsection}{0pc}{0em}{0pc}
\titlespacing{\subsubsection}{0pc}{0.33em}{0pc}
\titlespacing{\paragraph}{0em}{0.125em}{0.5em}
\setlength{\parindent}{2em}
\setlength{\parskip}{1em}
\linespread{1}

\renewcommand{\baselinestretch}{1.0}

\renewcommand\bf[1]{\textbf{#1}}
\renewcommand\it[1]{\textit{#1}}

\newcommand\ov[1]{\overline{#1}}
\newcommand{\vect}[1]{\mathbf{#1}}
\newcommand{\bb}[1]{\mathbb{#1}}
\newcommand{\vn}{\varnothing}
\newcommand\stk[2][black]{\setbox0=\hbox{$#2$}%
\rlap{\raisebox{.45\ht0}{\textcolor{#1}{\rule{\wd0}{1pt}}}}#2}

\makeatletter
\global\let\tikz@ensure@dollar@catcode=\relax
\makeatother

\makeatletter
\def\mcolor#1#{\@mcolor{#1}}
\def\@mcolor#1#2#3{%
  \protect\leavevmode
  \begingroup
    \color#1{#2}#3%
  \endgroup
}
\makeatother
\definecolor{notepadrule}{RGB}{217,244,244}

\backgroundsetup{
contents={%
  \begin{tikzpicture}
    \foreach \fila in {0,...,52}
    {
      \draw [line width=1pt,color=notepadrule]
      (current page.west|-0,-\fila*12pt) -- ++(\paperwidth,0);
    }
    \draw[overlay,red!70!black,line width=1pt]
      ([xshift=-1pt]current page text area.west|-current page.north) --
      ([xshift=-1pt]current page text area.west|-current page.south);
  \end{tikzpicture}%
},
scale=1,
angle=0,
opacity=1
}

\begin{document}

\setlength{\abovedisplayskip}{12pt}
\setlength{\belowdisplayskip}{12pt}
\setlength{\abovedisplayshortskip}{0pt}
\setlength{\belowdisplayshortskip}{0pt}
% \setlength{\baselineskip}{12pt}
\setlength{\jot}{2pt}

\section{Conceito de Função de Variáveis Aleatórias \& \it{Notação}}
Ideia Intuitiva: Conhecemos a variável aleatória $X$ e realizações de $X$. Aplicar uma função aos
resultados de uma v.a. gera novos resultados que correspondem a uma v.a. diferente. A pergunta
central é: se sabemos propriedades da v.a. original (e.g PDF, $\bb{E}, Var$), quais as propriedades
respectivas no segundo caso?

\subsection{Definição}
Seja $X: S \rightarrow \bb{R}$. Seja $g : \bb{R} \rightarrow \bb{R}$ uma função conhecida.
Define-se a variável aleatória $Y : S \rightarrow \bb{R}$ como sendo uma função de $X$ a partir da
seguinte relação:
\begin{align*}
    Y(s) = g(X(s)), \; \forall \; s \in S
\end{align*}
Observe que esta definição permite determinar uma função de probabilidade induzida:
\begin{align*}
    P_Y(A) = P_X(\{x \in \bb{R} / g(x) \in A\}), A \subseteq \bb{R}
\end{align*}
$P_Y$: satisfaz os axiomas de Kolmogorov.

\subsection{Notação}
A partir dessa definição, adota-se a notação: $Y = g(X)$. Observe que esta notação generaliza o
conceito de uma função numética, aplicando-a a uma variável aleatória. Generalizamos também o
conceito de número.

Um número, por exemplo 3, é uma variável aleatória com uma PDF que é um impulso: $f_X =
\delta(x-3)$, $\bb{E}(X) = 3$, $Var(X) = 0$.

$\bb{E}(X+5)$ denota o valor esperado de $X$ adicionado ao valor esperado de $5$ de tal forma que
$\bb{E}(X+5) = \bb{E}(X) + \bb{E}(5)$

\subsection{Breve questionamento sobre PDF de uma função de v.a.}
Como determinar a PDF de uma função de uma variável aleatória, se conhecemos a PDf da v.a.
original?

Um caminho para resolvermos esse problema é antes calcularmos a CDF de $Y$. Depois derivamos para
obter a PDF.
\begin{align*}
    \text{CDF de }Y: F_Y(y) = P_Y((-\infty, y]) \\
    F_Y(y) = P_X(\{x \in \bb{R}/ g(x) \in (-\infty,y]\}) \\
    F_Y(y) = P_X(\{x \in \bb{R}/ g(x) \leq y\}) \\
    f_Y(y) = \frac{d}{dy}F_Y(y)
\end{align*}
E como $f_Y(y) = \frac{d}{dy}F_Y(y)$ se relaciona com $f_X(x)$?

Consideremos uma caso particular mais simples:
\\
Considere uma v.a. $X$ de PDF conhecida $f_X$. Suponha que $g: \bb{R} \rightarrow \bb{R}$ é
estritamente crescente e inversível. Nesse caso:
\begin{align*}
    F_Y(y) = P_X(\{x \in \bb{R}/ g(x) \leq y\}) = P_X((-\infty, g^{-1}(y)])
\end{align*}

Temos: $P_X((-\infty,x]) = F_X(x)$ e $P_X((-\infty,g^{-1}(y)]) = P_X(g^{-1}(y))$ paro caso
inversível e estritamente crescente.

Calculando a derivada e fazendo $h(y) = g^{-1}(y)$:
\begin{align*}
    f_Y(y) = \frac{d}{dy}F_Y(y) \\
    F_Y(y) = F_X(h(y)) \\
    f_Y(y) = F_X'(h(y)) \cdot h'(y) \\
    f_Y(y) = f_X(h(y)) \cdot h'(y) \\
    \text{lembrando que: } h'(y) = \frac{1}{g'(h(y))} \\
    g(h(y)) = y \\
    g'(h(y)) \cdot h'(y) = 1 \\
    f_Y(y) = f_X(g^{-1}(y)) \cdot \frac{1}{g'(g^{-1}(y))} \\
    \\
    f_Y(y) = \frac{f_X(g^{-1}(y))}{g'(g^{-1}(y))}
\end{align*}

Sendo válida para $g$ inversível e que $g^{-1}(y) > 0 \; \forall \; y \in \bb{R}$

\subsubsection*{Exercício:}
Repita o problema considerando que $g$ é inversível e estritamente decrescente.


\end{document}
