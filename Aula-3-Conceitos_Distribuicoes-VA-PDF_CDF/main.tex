\documentclass{article}
\usepackage{enumitem}
\usepackage[T1]{fontenc}
\usepackage[utf8]{inputenc}
\usepackage[portuguese]{babel}
\usepackage[vmargin=3cm]{geometry}
\usepackage{tikzpagenodes}
\usepackage{lipsum}
\usepackage{xcolor}
\usepackage{cancel}
\usepackage{amsmath}
\usepackage{amssymb}
\usepackage{background}
\usepackage{titlesec}
\usepackage[nodisplayskipstretch]{setspace}
\usepackage{hyphenat}
\usepackage[normalem]{ulem}
\hyphenation{mate-mática recu-perar}

\titlespacing{\section}{0pc}{-0.5em}{0pc}
\titlespacing{\subsection}{0pc}{0em}{0pc}
\titlespacing{\subsubsection}{0pc}{0.33em}{0pc}
\titlespacing{\paragraph}{0em}{0.125em}{0.5em}
\setlength{\parindent}{2em}
\setlength{\parskip}{1em}
\linespread{1}

\renewcommand{\baselinestretch}{1.0}

\renewcommand\bf[1]{\textbf{#1}}
\renewcommand\it[1]{\textit{#1}}

\newcommand\ov[1]{\overline{#1}}
\newcommand{\vect}[1]{\mathbf{#1}}
\newcommand{\vn}{\varnothing}
\newcommand\stk[2][black]{\setbox0=\hbox{$#2$}%
\rlap{\raisebox{.45\ht0}{\textcolor{#1}{\rule{\wd0}{1pt}}}}#2}

\makeatletter
\def\mcolor#1#{\@mcolor{#1}}
\def\@mcolor#1#2#3{%
  \protect\leavevmode
  \begingroup
    \color#1{#2}#3%
  \endgroup
}
\makeatother
\definecolor{notepadrule}{RGB}{217,244,244}

\backgroundsetup{
contents={%
  \begin{tikzpicture}
    \foreach \fila in {0,...,52}
    {
      \draw [line width=1pt,color=notepadrule]
      (current page.west|-0,-\fila*12pt) -- ++(\paperwidth,0);
    }
    \draw[overlay,red!70!black,line width=1pt]
      ([xshift=-1pt]current page text area.west|-current page.north) --
      ([xshift=-1pt]current page text area.west|-current page.south);
  \end{tikzpicture}%
},
scale=1,
angle=0,
opacity=1
}

\begin{document}

\setlength{\abovedisplayskip}{12pt}
\setlength{\belowdisplayskip}{12pt}
\setlength{\abovedisplayshortskip}{0pt}
\setlength{\belowdisplayshortskip}{0pt}
% \setlength{\baselineskip}{12pt}
\setlength{\jot}{0pt}

\section{Experimento de Bernoulli}
O experimento é bastante simples e apresenta um espaço amostral $S$ com $|S| = 2$. Podemos denotar
$S = \{s_1,s_2\}$. Ademais,

\begin{itemize}[noitemsep,topsep=0pt]
    \item $P(\{s_1\}) = p$, $p \in [0,1]$
    \item $P(\{s_2\}) = 1 - p$
    \item $P(\vn) = 0$
    \item $P(S) = 1$
\end{itemize}
Satisfaz os axiomas de Kolmogorov independente doo valor de $p$.
\\
\section{Experimento Binomial}
$S \rightarrow$ conjunto de ênuplas ordenadas, em cada elemento de cada ênupla assume 1 de 2
valores, com os \bf{eventos associados ao caso 1 ou caso 2 em cada posição} sendo
\it{independentes} de uma posição para outra (o evento associado ao i-ésimo termo de ser $s_1$
independe do j-ésimo termo ser $s_1, \; \forall \; i,j \in \{1,2,\ldots,n\} \text{ com } i \neq
j$).

Nesta situação, nós provamos na última aula [vide notas da aula 2] que as probabilidades de eventos
associados a $k$ posições assumirem um dos dois valores (digamos $s_1$) seguem a chamada
distribuição binomial:
\begin{align*}
P(k \text{ elementos } 1) = {N \choose k} p^k (1-p)^{N-k}, \text{todas as ênuplas com k elementos iguais a 1}
\end{align*}

\vspace{-0.5em}
\section{Variáveis Aleatórias}
\bf{Conceito Formal:} Uma variável aleatória $X$ é uma função que associa o resultado de um
experimento a um número, que pode ser real, complexo, inteiro, fracionário etc;

Uma variável $X: S \rightarrow \mathbb{D}$ é dita contínua se $C \mathbb{D}$ é um conjunto numérico
incontável; é dita discreta se $C \mathbb{D}$ é um conjunto contável.

Dado um experimento com espaço amostral $S$:
\begin{align*}
    X: S \rightarrow \mathbb{R} \text{ ou } X: S \rightarrow \mathbb{C} \text{ ou } X: S
    \rightarrow \mathbb{Z} \text{ etc}
\end{align*}

\vspace{-0.75em}
\subsection{Probabilidade Induzida - $P: \varepsilon \rightarrow [0,1]$}
Considere um conjunto amostral $S$ vinculado a um espaço de probabilidades ($S, \varepsilon, P$).
Considere ainda um v.a. $X: S \rightarrow \mathbb{R}$. Podemos definir a probabilidade induzida por
$X$ como sendo:
\begin{align*}
    P_X(\mathbb{A}) = P(\{s \in S / X(s) \in \mathbb{A}\}), \text{onde } \mathbb{A} \in \varepsilon_{\mathbb{R}}, \text{ com } \varepsilon_{\mathbb{R}} \subseteq
    P(\mathbb{R})
\end{align*}
Podemos provar que $P_X$ satisfaz os axiomas de Kolmogorov.

\paragraph{Exemplo:}\mbox{}\\
$S:$ conjunto de ênuplas ordenadas em um experimento binomial.\\
$X(s):$ número de elementos na ênupla s iguais ao valor 1.\\
$P(\{k\}) = {N \choose k} p^k (1-p)^{N-k}$

\paragraph{Notação:}
\begin{align*}
\stk[red]{P_X(2), P_X(3)} &\rightarrow P_X(\{2\}), P_X(\{3\})\\
\stk[red]{P_X(2,3)} &\rightarrow P_X(\{2,3\}) = {n \choose 2}p^2(1-p)^{n-2} + {n \choose 3}p^3(1-p)^{n-3} \\
P(X \in [2,4]) &= P(\{s \in S / X(s) \in [2,4]\}) = P_X([2,4]) \\
P(X^2 + X^3 > 2 e X^2 + X^3 < 3) &= P(\{s \in S / X(s) \in (2,3)\}) \\
P(|X-\mu_X| > k \sigma) &= P(\{s \in S / |X(s) - \mu_X| > k \sigma \}) = P_X(\mathbb{A})
\end{align*}

\section{Função Densidade de Probabilidade - PDF}
Intuitivamente, a função densidade de probabilidade (PDF) é uma função que, ao ser integrada em um
conjunto numérico, fornece a probabilidade desse conjunto.

\subsection{Definição formal (pré-requisito: CDF)}
Seja $X: S \rightarrow \mathbb{R}$ uma variável aleatória. A função de distribuição acumulada de
probabilidade (CDF) de $X$ é a função $F_X: \mathbb{R} \rightarrow [0,1]$ definida por:
\begin{align*}
    F_X(x) = P_X([-\infty, x]) = P(s \in S / X(s) \leq x)
\end{align*}

\paragraph{Exemplo:}
Experimento de lançar um dado de seis faces. Dado equilibrado (iid). $S = \{a,b,c,d,e,f\}$\\
$S \rightarrow \mathbb{R} \; X(a) = 1;\; X(b) = 2;\; X(c) = 3;\; \ldots X(f) = 6$
\begin{align*}
    F_X(0) = 0 && F_X(0.9) = 0 && F_X(1) = 1/6 && F_X(1.5) = 1/6 \\
    F_X(1.98) = 1/6 && F_X(2) = 2/6 && F_X(2.9) = 2/6 && F_X(3) = 3/6 \\
    \ldots
\end{align*}

\paragraph{Finalmente:}
$P_X((2,5]) = F_X(5) - F_X(2) = F_X((-\infty, 5]) - F_X((-\infty, 2])$

\end{document}
