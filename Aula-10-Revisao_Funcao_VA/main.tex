\documentclass{article}
\usepackage{enumitem}
\usepackage[T1]{fontenc}
\usepackage[utf8]{inputenc}
\usepackage[portuguese]{babel}
\usepackage[vmargin=3cm]{geometry}
\usepackage{tikzpagenodes}
\usepackage{lipsum}
\usepackage{xcolor}
\usepackage{cancel}
\usepackage{amsmath}
\usepackage{mathrsfs}
\usepackage{amssymb}
\usepackage{background}
\usepackage{titlesec}
\usepackage[nodisplayskipstretch]{setspace}
\usepackage{hyphenat}
\usepackage[normalem]{ulem}
\usepackage{minted}
\usepackage{subcaption}
\hyphenation{mate-mática recu-perar}

\titlespacing{\section}{0pc}{-0.25em}{0pc}
\titlespacing{\subsection}{0pc}{0em}{0pc}
\titlespacing{\subsubsection}{0pc}{0.33em}{0pc}
\titlespacing{\paragraph}{0em}{0.25em}{0.5em}
\setlength{\parindent}{2em}
\setlength{\parskip}{1em}
\linespread{1}

\renewcommand{\baselinestretch}{1.0}

\renewcommand\bf[1]{\textbf{#1}}
\renewcommand\it[1]{\textit{#1}}

\newcommand\ov[1]{\overline{#1}}
\newcommand{\vect}[1]{\mathbf{#1}}
\newcommand{\bb}[1]{\mathbb{#1}}
\newcommand{\vn}{\varnothing}
\newcommand\stk[2][black]{\setbox0=\hbox{$#2$}%
\rlap{\raisebox{.45\ht0}{\textcolor{#1}{\rule{\wd0}{1pt}}}}#2}

\makeatletter
\global\let\tikz@ensure@dollar@catcode=\relax
\makeatother

\makeatletter
\def\mcolor#1#{\@mcolor{#1}}
\def\@mcolor#1#2#3{%
  \protect\leavevmode
  \begingroup
    \color#1{#2}#3%
  \endgroup
}
\makeatother
\definecolor{notepadrule}{RGB}{217,244,244}

\backgroundsetup{
contents={%
  \begin{tikzpicture}
    \foreach \fila in {0,...,52}
    {
      \draw [line width=1pt,color=notepadrule]
      (current page.west|-0,-\fila*12pt) -- ++(\paperwidth,0);
    }
    \draw[overlay,red!70!black,line width=1pt]
      ([xshift=-1pt]current page text area.west|-current page.north) --
      ([xshift=-1pt]current page text area.west|-current page.south);
  \end{tikzpicture}%
},
scale=1,
angle=0,
opacity=1
}

\begin{document}

\setlength{\abovedisplayskip}{12pt}
\setlength{\belowdisplayskip}{12pt}
\setlength{\abovedisplayshortskip}{0pt}
\setlength{\belowdisplayshortskip}{0pt}
% \setlength{\baselineskip}{12pt}
\setlength{\jot}{1pt}

\section{Breve Revisão}
Considere uma v.a. $X \sim \mathscr{U}([-2,2))$. Considere ainda $g(x) = x^2, \; \forall x \in
\bb{R}$. Qual o valor esperado de $Y = g(X)$? Qual a PDF de $Y = g(X)$ ?

Temos:
\begin{align*}
    f_X(x)&=
        \begin{cases}
            \frac{1}{4}, \; \forall \; x \in [-2,2) \\
            0, \text{ caso contrário}
        \end{cases}
        \\
    f_Y(y)&=
        \begin{cases}
            0, \text{ se } x < 0 \\
            0, \text{ se } x > 4 \\
            \text{(?)}, \text{ se } 0 \leq x \leq 4 \\
        \end{cases}
        \\ \\
    E(y) &= \int^\infty_{-\infty} y \cdot f_y(y)dy = \mcolor{blue}{\int^\infty_{-\infty} g(x) \cdot
    f_X(x)dx} \\
        E(y) &= \int^2_{-2} x^2 \cdot \frac{1}{4} dx = \frac{4}{3} \\[-1em]
\end{align*}
Para todo $y \in (0,4)$:

$x_1 = - \sqrt{y} \text{ e } x_2 = \sqrt{y} \rightarrow x_1^2 = x_2^2 = y$
\begin{align*}
    f_Y(y) &= \sum_i \frac{f_X(x_i)}{|g'(x_i)|} = \frac{f_X(\sqrt{y})}{|2\sqrt{y}|} +
    \frac{f_X(-\sqrt{y})}{|-2\sqrt{y}|}\\
    f_Y(y) &= 2 \cdot \frac{0.25}{2\sqrt{y}} = \frac{0.25}{\sqrt{y}} \; \forall \; y \in (0,4) \\
    f_Y(y)&=
    \begin{cases}
        0, \text{ se } x < 0 \\
        0, \text{ se } x > 4 \\
        \frac{0.25}{\sqrt{y}}, \text{ se } 0 \leq x \leq 4 \\
    \end{cases}
\end{align*}
\\[-0.75em]
\section{Um outro exercício para casa...}
Considere $X$ uma variável aleatória com PDF $f_X$ e CDF $F_X(x)$. Considere ainda $U \sim
\mathscr{U}([0,1))$. Defina $g:\bb{R} \rightarrow \bb{R}$ tal que $g(u) = F^{-1}(u)$ ($F$ é suposta
inversível). Qual a PDF de $Y=g(U)$?

\section{Observação sobre notação de Momentos de V.A.}
Definimos o momento central de uma variável aleatória $X$ como sendo:
\begin{align*}
    M_C^n &= \int_{-\infty}^\infty (x - \mu)^n f_X(x)dx \\
          &\text{Se definirmos } Y = g(X), \text{ com } g: \bb{R} \rightarrow \bb{R} \text{ é dado
          por } g(x) = (x - \mu)^n \\ \\
    E(y) &= \int^\infty_{-\infty} y \cdot f_y(y)dy = \mcolor{blue}{\int^\infty_{-\infty} g(x) \cdot
    f_X(x)dx} \\ \\
         &\text{Logo: } M_C^n = E((x - \mu)^n) \\
         &\text{Da mesma forma: } M^n = E(X^n) \\ \\
         &\text{Ademais, para a variância:} \\
        \text{Var} &= \sigma^2 = \int^{\infty}_{-\infty} (x-\mu)^2 f_X(x)dx = E((x - \mu)^2)
\end{align*}



\end{document}
