\documentclass{article}
\usepackage{enumitem}
\usepackage[T1]{fontenc}
\usepackage[utf8]{inputenc}
\usepackage[portuguese]{babel}
\usepackage[vmargin=3cm]{geometry}
\usepackage{tikzpagenodes}
\usepackage{lipsum}
\usepackage{xcolor}
\usepackage{cancel}
\usepackage{amsmath}
\usepackage{mathrsfs}
\usepackage{amssymb}
\usepackage{background}
\usepackage{titlesec}
\usepackage[nodisplayskipstretch]{setspace}
\usepackage{hyphenat}
\usepackage[normalem]{ulem}
\usepackage{subcaption}
\hyphenation{mate-mática recu-perar}

\titlespacing{\section}{0pc}{-0.25em}{0pc}
\titlespacing{\subsection}{0pc}{0em}{0pc}
\titlespacing{\subsubsection}{0pc}{0.33em}{0pc}
\titlespacing{\paragraph}{0em}{0.25em}{0.5em}
\setlength{\parindent}{2em}
\setlength{\parskip}{1em}
\linespread{1}

\renewcommand{\baselinestretch}{1.0}

\renewcommand\bf[1]{\textbf{#1}}
\renewcommand\it[1]{\textit{#1}}

\newcommand\ov[1]{\overline{#1}}
\newcommand{\vect}[1]{\mathbf{#1}}
\newcommand{\bb}[1]{\mathbb{#1}}
\newcommand{\vn}{\varnothing}
\newcommand\stk[2][black]{\setbox0=\hbox{$#2$}%
\rlap{\raisebox{.45\ht0}{\textcolor{#1}{\rule{\wd0}{1pt}}}}#2}

\makeatletter
\global\let\tikz@ensure@dollar@catcode=\relax
\makeatother

\makeatletter
\def\mcolor#1#{\@mcolor{#1}}
\def\@mcolor#1#2#3{%
  \protect\leavevmode
  \begingroup
    \color#1{#2}#3%
  \endgroup
}
\makeatother
\definecolor{notepadrule}{RGB}{217,244,244}

\backgroundsetup{
contents={%
  \begin{tikzpicture}
    \foreach \fila in {0,...,52}
    {
      \draw [line width=1pt,color=notepadrule]
      (current page.west|-0,-\fila*12pt) -- ++(\paperwidth,0);
    }
    \draw[overlay,red!70!black,line width=1pt]
      ([xshift=-1pt]current page text area.west|-current page.north) --
      ([xshift=-1pt]current page text area.west|-current page.south);
  \end{tikzpicture}%
},
scale=1,
angle=0,
opacity=1
}

\begin{document}

\setlength{\abovedisplayskip}{12pt}
\setlength{\belowdisplayskip}{12pt}
\setlength{\abovedisplayshortskip}{0pt}
\setlength{\belowdisplayshortskip}{0pt}
% \setlength{\baselineskip}{12pt}
\setlength{\jot}{1pt}

\section{Vetores Aleatórios - PDF Conjunta}
Lembrando do caso de uma variável aleatória no qual:
\begin{align*}
    &\int^{b}_{a} f_X(x)dx = P_X((a,b]) \\ \\
    &\text{Para 3 variáveis:} \\
    &\int \int \int_{V} f_X(x_1,x_2,x_3) d x_1 d x_2 d x_3 = P_X(V), \\
    &\text{no qual V é um conjunto de pontos em } \mathbb{R}^3
.\end{align*}
\\[0.5em]
Para definirmos formalmente a PDF precisamos do conceito de CDF.

\subsection{CDF de um Vetor Aleatório}
Podemos fazer:
\begin{align*}
    F_X(x_1,x_2,x_3,\ldots,x_N) = P_X((-\infty,x_1] \times (-\infty,x_2] \times \ldots
    (-\infty,x_N]) \\
    \text{com a expansão para: } \{1,2\} \times \{3,4\} = \{(1,3), (1,4), (2,3), (2,4)\} \\
    \text{o operador $\times$ é o produto cartesiano}
\end{align*}

\subsection{Definição da PDF Conjunta}
É a função obtida pela derivação parcial da CDF com respeito a todas as componentes do vetor. No
exemplo em que $F_X(x_1,x_2)=P_X(A)$, temos $f_X(x_1,x_2) = \frac{\partial^{2}}{\partial x_1
x_2}F_X(x_1,x_2)$.
\begin{align*}
    f_X(x_1, x_2, \ldots, x_N) = \frac{\partial^{N}}{\partial x_1 \partial x_2 \ldots \partial x_N
    }F_X(x_1, x_2, \ldots, x_N)
.\end{align*}
\\
\section{PDF Marginal de um Vetor Aleatório}
Considere um vetor aleatório $X$ de $N$ componentes. Considere ainda uma função $g: \mathbb{R}^N
\to \mathbb{R}$, tal que $g(x)$ é i-ésima componente de $X$, com $1 \leq i \leq N$ com $i$
inteiro.

Temos uma variável aleatória $Y = g(X)$, definido por $Y(s) = g(X(s))$. A PDF de $X$ com respeito a
i-ésima componente é definida como a PDF de $Y$. Denominada PDF Marginal de $X$ na i-ésima
componente.

\bf{Pergunta:} se temos a PDF conjunta de $X$, como calcularmos a PDF marginal com respeito à
i-ésima componente?
\begin{align*}
    f_{X_i}(x_i) = \underbrace{\int \int}_{N-1 \text{ integrais}} f_X(x_1, x_2, \ldots, x_N)
    \underbrace{d x_2 d x_{i-1} d x_{i+1} d x_{i+2} d x_N}_{dx_j \; \forall j \in \{1,\ldots,N\} -
\{i\}}
.\end{align*}

Extensão do conceito da PDF Marginal para m componentes do vetor $X$ difere em definirmos a
seguinte função de vetor aleatório:
\begin{align*}
    g: \mathbb{R}^{N} \to \mathbb{R}^{m}, m \le N,
.\end{align*}
com g(x) o vetor formado pelas $m$ componentes escolhidas de $Y  = g(X)$ é um vetor aleatório
de $m$ componentes.

A PDF conjunta de $Y$ é uma PDF Marginal de $X$:
\begin{align*}
    f_{X_1, X_2, \ldots, X_M}(x_1, x_2, x_3, \ldots, x_m) = \underbrace{\int \int}_{N-m \text{
    integrais}} f_X(x_1, x_2, \ldots, x_N) \underbrace{d x_2 d x_{i-1} d x_{i+1} d x_{i+2} d x_N}_{\text{N-m componentes de
    integração}}
.\end{align*}

\bf{Observação:} Sabendo a PDF conjunta de um vetor, conseguimos calcular a PDF marginal com
respeito a quaisquer grupo de componentes (integrando com respeito a todos os outros
componentes). O contrário nem sempre é possível (calcular a conjunta a partir de marginais).
\\[0.5em]
\subsection*{Breve Exemplo}
Considere $X$ um vetor aleatório $X: S \to \mathbb{R}^{2}$ e $f_x(x_1, x_2) = x_1 + x_1 x_2, \forall
x_1, x_2 \in \mathbb{R}^2$ tais que $0 \le x_1 \le \frac{1}{2}, 0 \le x_2 \le K$, com $K \in
\mathbb{R}$ e $f_X(x_1, x_2) = 0$, caso contrário.

\bf{Ponto Inicial:} Determinar o valor de $K$ para definir a PDF Conjunta.

\paragraph{A} Qual a PDF Marginal de $X_1$?
\\
Integrar $X_2$ de $0 \to K$

\paragraph{B} Qual a PDF Marginal de $X_2$?
\\
Integrar $X_1$ de $0 \to 0.5$

\paragraph{C} Qual a probabilidade de $X_1 > X_2$?
\\
Considerar a região total em que $X_1 > X_2$ (região abaixo da reta $y=x$) e integrar para obter
probabilidade.

\paragraph{D} Qual a probabilidade de $X_2$ estar entre $\frac{K}{2}$ e $K$?
\\
É possível usar a PDF marginal nesse caso.

\paragraph{E} $X_1 \text{ e } X_2$ são dependentes ou independentes?
\\
Essa é \it{spoiler} de aulas posteriores.

\end{document}
