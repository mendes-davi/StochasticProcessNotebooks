\documentclass{article}
\usepackage{enumitem}
\usepackage[T1]{fontenc}
\usepackage[utf8]{inputenc}
\usepackage[portuguese]{babel}
\usepackage[vmargin=3cm]{geometry}
\usepackage{tikzpagenodes}
\usepackage{lipsum}
\usepackage{xcolor}
\usepackage{cancel}
\usepackage{amsmath}
\usepackage{mathrsfs}
\usepackage{amssymb}
\usepackage{background}
\usepackage{titlesec}
\usepackage[nodisplayskipstretch]{setspace}
\usepackage{hyphenat}
\usepackage[normalem]{ulem}
\usepackage{subcaption}
\hyphenation{mate-mática recu-perar}

\titlespacing{\section}{0pc}{-0.25em}{0pc}
\titlespacing{\subsection}{0pc}{0em}{0pc}
\titlespacing{\subsubsection}{0pc}{0.33em}{0pc}
\titlespacing{\paragraph}{0em}{0.25em}{0.5em}
\setlength{\parindent}{2em}
\setlength{\parskip}{1em}
\linespread{1}

\renewcommand{\baselinestretch}{1.0}

\renewcommand\bf[1]{\textbf{#1}}
\renewcommand\it[1]{\textit{#1}}

\newcommand\ov[1]{\overline{#1}}
\renewcommand\u[1]{\underbar{#1}}
\newcommand{\vect}[1]{\mathbf{#1}}
\newcommand{\bb}[1]{\mathbb{#1}}
\newcommand{\vn}{\varnothing}
\newcommand\stk[2][black]{\setbox0=\hbox{$#2$}%
\rlap{\raisebox{.45\ht0}{\textcolor{#1}{\rule{\wd0}{1pt}}}}#2}

\makeatletter
\global\let\tikz@ensure@dollar@catcode=\relax
\makeatother

\makeatletter
\def\mcolor#1#{\@mcolor{#1}}
\def\@mcolor#1#2#3{%
  \protect\leavevmode
  \begingroup
    \color#1{#2}#3%
  \endgroup
}
\makeatother
\definecolor{notepadrule}{RGB}{217,244,244}

\backgroundsetup{
contents={%
  \begin{tikzpicture}
    \foreach \fila in {0,...,52}
    {
      \draw [line width=1pt,color=notepadrule]
      (current page.west|-0,-\fila*12pt) -- ++(\paperwidth,0);
    }
    \draw[overlay,red!70!black,line width=1pt]
      ([xshift=-1pt]current page text area.west|-current page.north) --
      ([xshift=-1pt]current page text area.west|-current page.south);
  \end{tikzpicture}%
},
scale=1,
angle=0,
opacity=1
}

\begin{document}

\setlength{\abovedisplayskip}{12pt}
\setlength{\belowdisplayskip}{12pt}
\setlength{\abovedisplayshortskip}{0pt}
\setlength{\belowdisplayshortskip}{0pt}
% \setlength{\baselineskip}{12pt}
\setlength{\jot}{1pt}

\section{Descrições Estatísticas de Processos Estocásticos}
As principais descrições possíveis são: PDFs e Momentos. Considere um $t$ fixo; $X_{(t)}$ pode ser
tratado como uma variável aleatória induzida. Nesse instante, podemos definir a PDF da v.a.
induzida como $f_{X_C(t)}(x)$ ou $f_{X_C}(t, x)$: \textit{uma PDF marginal do processo para um
instante}. Note que não é um descrição completa do processo, pois não reflete as possíveis
dependências entre as v.a. associadas a instantes diferentes.

Descrição completa do processo estocástico:
\begin{itemize}
    \item PDFs conjuntas associadas a todas as combinações de N instantes, para todo N inteiro
        positivo
    \item Todas as PDFs da forma $f_{X_C}(t_1,x_1;t_2,x_2;\ldots;t_N,x_N)$ para a qual é definida
        uma PDF um vetor aleatório (com N pontos fixados)
        \begin{itemize}
            \item Note: a escolha de pontos $t_1,t_2,\ldots,t_N$, induz um vetor aleatório para o
                qual a expressão acima é a PDF conjunta desse vetor aleatório.
            \item Temos que conhecer para qualquer escolha de $t_1,t_2,\ldots,t_N$ para dizermos
                que temos uma descrição completa.
        \end{itemize}
\end{itemize}

\subsection{Momentos de um Processo Estocásticos}
Podemos medir momentos associados um processo aleatório.
\begin{itemize}
    \item Podemos definir todos os momentos do vetor aleatório induzido por um escolha de instantes
        $t_1,t_2,\ldots,t_N$.
        \begin{itemize}
            \item Vetor média
            \item Vetor de variâncias
            \item Matriz de covariâncias
            \item Matriz de correlação
        \end{itemize}
    \item Podemos definir uma função média (função expectâncias da variável induzida em instante
        escolhido $t$)
    \item Temos ainda uma função de autocovariância (autocovariância associada as componentes do
    \item Por fim, temos a ainda a função de autocorrelação
        vetor induzido pela escolha desses dois pontos)
        \begin{itemize}
            \item $\mu(t) = \int_{-\infty}^{\infty} x \cdot f_{X_C}(t,x)dx $
            \item $C(t_1, t_2) = \int_{-\infty}^{\infty}\int_{-\infty}^{\infty}
                (x_1-\mu(t_1))(x_2-\mu(t_2))^* \cdot f_{X_C}(t_1,x_1,t_2,x_2) dx_1 dx_2$
            \item $R(t_1, t_2) = \int_{-\infty}^{\infty}\int_{-\infty}^{\infty} x_1 \cdot x_2^* \cdot f_{X_C}(t_1,x_1,t_2,x_2) dx_1 dx_2$
        \end{itemize}
\end{itemize}

\newpage
\section{Processos Estacionários no Sentido Estrito e Amplo}
Um processo estocástico é dito estacionário \textbf{no sentido estrito} se e somente se:
\begin{itemize}
    \item A PDF conjunta do vetor induzido por uma escolha de $N$ instantes depende apenas dos
        intervalos entre esses instantes considerados, sendo uma condição válida para qualquer
        quantidade de N pontos $\to N \in \{1,2,3,4, \ldots \}$.
\end{itemize}

O que pode ser interpretado como: A PDF é igual desde que sejam mantidas as distâncias entre os
instantes observados.
\begin{align*}
    f_{X_C}(t_1,x_1,t_2,x_2,t_3,x_3,\ldots,t_N,x_N) = f_{X_C}(t_1+\Delta t,x_1,t_2+\Delta
    t,x_2,t_3+\Delta t,x_3,\ldots,t_N+\Delta t,x_N), \forall \Delta t \in \mathbb{R}
\end{align*}

Um processo estocástico é dito estacionário \textbf{no sentido amplo} (critério menos exigente que
para o sentido estrito) se e somente se:

\begin{itemize}
    \item A função expectância é constante (independente de $t$)
    \item A função de correlações só depende da diferença entre os instantes considerados: $R(t_1,
        t_2) = R(t_1 + \Delta t, t_2 + \Delta t), \forall t_1, t_2, \Delta t$
\end{itemize}

Note que, se o processo é estacionário no sentido amplo:
\begin{align*}
    C(t_1, t_2) &= R(t_1,t_2) - \mu(t_1) \cdot \mu(t_2)^* \\
    C(t_1 + \Delta t, t_2 + \Delta t) &= R(t_1 + \Delta t, t_2 + \Delta t) - \mu(t_1+\Delta t) \cdot
    \mu(t_2 + \Delta t)^* \\
    C(t_1 + \Delta t, t_2 + \Delta t) &= R(t_1, t_2) - \mu(t_1) \cdot \mu(t_2)^* \\
    C(t_1, t_2) &= C(t_1 + \Delta t, t_2 + \Delta t)
\end{align*}

\subsection{Notação}
No caso de processos estacionário no sentido amplo, a covariância e correlação não precisam mais ser
definidas em função de dois pontos. Podemos definir as funções em termos de duas funções de uma
única variável, que é a distância entre os dois pontos considerados.
\begin{align*}
    r(t) &= R(t_1, t_1+t) \text{ o resultado é o mesmo para qualquer } t \\
    c(t) &= C(t_1, t_1+t) \\
\end{align*}



\end{document}
