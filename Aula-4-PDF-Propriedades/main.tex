\documentclass{article}
\usepackage{enumitem}
\usepackage[T1]{fontenc}
\usepackage[utf8]{inputenc}
\usepackage[portuguese]{babel}
\usepackage[vmargin=3cm]{geometry}
\usepackage{tikzpagenodes}
\usepackage{lipsum}
\usepackage{xcolor}
\usepackage{cancel}
\usepackage{amsmath}
\usepackage{amssymb}
\usepackage{background}
\usepackage{titlesec}
\usepackage[nodisplayskipstretch]{setspace}
\usepackage{hyphenat}
\usepackage[normalem]{ulem}
\hyphenation{mate-mática recu-perar}

\titlespacing{\section}{0pc}{-0.5em}{0pc}
\titlespacing{\subsection}{0pc}{0em}{0pc}
\titlespacing{\subsubsection}{0pc}{0.33em}{0pc}
\titlespacing{\paragraph}{0em}{0.125em}{0.5em}
\setlength{\parindent}{2em}
\setlength{\parskip}{1em}
\linespread{1}

\renewcommand{\baselinestretch}{1.0}

\renewcommand\bf[1]{\textbf{#1}}
\renewcommand\it[1]{\textit{#1}}

\newcommand\ov[1]{\overline{#1}}
\newcommand{\vect}[1]{\mathbf{#1}}
\newcommand{\vn}{\varnothing}
\newcommand\stk[2][black]{\setbox0=\hbox{$#2$}%
\rlap{\raisebox{.45\ht0}{\textcolor{#1}{\rule{\wd0}{1pt}}}}#2}

\makeatletter
\def\mcolor#1#{\@mcolor{#1}}
\def\@mcolor#1#2#3{%
  \protect\leavevmode
  \begingroup
    \color#1{#2}#3%
  \endgroup
}
\makeatother
\definecolor{notepadrule}{RGB}{217,244,244}

\backgroundsetup{
contents={%
  \begin{tikzpicture}
    \foreach \fila in {0,...,52}
    {
      \draw [line width=1pt,color=notepadrule]
      (current page.west|-0,-\fila*12pt) -- ++(\paperwidth,0);
    }
    \draw[overlay,red!70!black,line width=1pt]
      ([xshift=-1pt]current page text area.west|-current page.north) --
      ([xshift=-1pt]current page text area.west|-current page.south);
  \end{tikzpicture}%
},
scale=1,
angle=0,
opacity=1
}

\begin{document}

\setlength{\abovedisplayskip}{12pt}
\setlength{\belowdisplayskip}{12pt}
\setlength{\abovedisplayshortskip}{0pt}
\setlength{\belowdisplayshortskip}{0pt}
% \setlength{\baselineskip}{12pt}
\setlength{\jot}{0pt}

\section{Função Densidade de Probabilidade - PDF}
Intuitivamente, a função densidade de probabilidade (PDF) é uma função que, ao ser integrada em um
conjunto numérico, fornece a probabilidade desse conjunto.

\subsection{Definição formal (pré-requisito: CDF)}
Seja $X: S \rightarrow \mathbb{R}$ uma variável aleatória. A função de distribuição acumulada de
probabilidade (CDF) de $X$ é a função $F_X: \mathbb{R} \rightarrow [0,1]$ definida por:
\begin{align*}
    F_X(x) = P_X([-\infty, x]) = P(s \in S / X(s) \leq x)
\end{align*}

\subsection*{Objetivo \& Definição:}

Como queremos $\int^b_a f_X(x)dx = P_X((a,b]) = F_X(b) - F_X(a)$.

Para que isso aconteça, vamos definir $f_X(x) = \frac{d}{dx}F_X(x)$

\textbf{Definição:} Dada uma variável aleatória com função de distribuição acumulada $F_x$, a
função densidade de probabilidade (PDF) é definida como:
\begin{align*}
f_X(x) &= \frac{d}{dx}F_X(x), \\
       &\text{e fica valendo a propriedade:} \\
\int^b_a f_X(x)dx &= P_X((a,b]) = F_X(b) - F_X(a)
\end{align*}
\\[-1.5em]
\section{Relação entre PDF e Função Massa de Probabilidade (PMF)}
Considere uma variável discreta $\bf{x}$, ou seja, dado o espaço amostral $S$, a imagem de $X: S
\rightarrow C \mathbb{D}_X$ é um conjunto discreto (finito ou infinito contável).
\\
\bf{Exemplo de uma v.a. discreta:} jogar um dado de 6 faces $I_X = \{1,2,3,4,5,6\}$ (conjunto
discreto)
\\
\bf{Exemplo de uma v.a. contínua:} Aferir a pressão sistólica de um indivíduo $I_X=\{0,200mHg\}$ (conjunto contínuo)
\\
Para a PMF, em cada conjunto podemos estabelecer a probabilidade de cada subconjunto unitário:
\begin{itemize}[noitemsep,topsep=2pt]
    \item $P_x(\{1\}) = 1/6$
    \item $P_x(\{2\}) = 1/6$
    \item $P_x(\{3\}) = 1/6$
    \item $P_x(\{4\}) = 1/6$
    \item $P_x(\{5\}) = 1/6$
    \item $P_x(\{6\}) = 1/6$
\end{itemize}

Para uma v.a. discreta podemos definir a PMF: função em domínio discreto que atribui a cada ponto
discreto de $I_X$ o valor de probabilidade do conjunto unitário com aquele ponto.
\\
Mas podemos calcular também PDF: função em domínio contínuo que, neste caso, vale 0 em todos os
pontos do domínio com exceção dos pontos que fazem parte da imagem de $\bf{x}$. Nesses pontos a
PDF é um impulso cuja magnitude é a probabilidade do conjunto em cada ponto apenas.

\section{Propriedades da Função de Densidade Probabilidade - PDF}
Essencialmente,
\begin{align*}
    f_X(x) &\geq 0, \; \forall \; x \in \mathbb{R} \\
    \int^\infty_{-\infty} f_X(x)dx &= 1
\end{align*}
\bf{Dada qualquer função $g$ que satisfaça as duas propriedades acima uma existe variável aleatória
cuja PDF é $g$.}
\\
Na prática, na definição de PDFs desconsideramos descontinuidades em pontos discretos, a favor de
versões contínuas naqueles pontos. \it{vide Teorema da Inversão}

\subsection{Algumas Propriedades \ldots}
Dada uma v.a. $X$ com PDF $f_X(x)$, alguns números descrevem as estatística básica de $X$. Em ordem
de importância:

$E(x) \rightarrow$ valor esperado, média, expectância, esperança matemática
\begin{align*}
    E(X) = \int^\infty_{-\infty} x \cdot f_X(x)dx \rightarrow \text{média dos valores, ponderada pela PDF de cada valor.}
\end{align*}
\\[-0.25em]
Variância: Média de todos os desvios quadráticos, ponderados pela PDF dos valores.
\begin{align*}
    Var(X) = \int^\infty_{-\infty} (x - E(X))^2 \cdot f_X(x)dx
\end{align*}

Desvio padrão $\sigma$: $\sigma - \sqrt{Var(X)} \rightarrow$ envolve uma ideia de variação mas
mantém as unidades da v.a.


\end{document}
