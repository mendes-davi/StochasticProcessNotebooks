\documentclass{article}
\usepackage[T1]{fontenc}
\usepackage[utf8]{inputenc}
\usepackage[portuguese]{babel}
\usepackage[vmargin=3cm]{geometry}
\usepackage{tikzpagenodes}
\usepackage{lipsum}
\usepackage{xcolor}
\usepackage{hyperref}
\usepackage{amsmath}
\usepackage{amssymb}
\usepackage{background}
\usepackage{titlesec}
\usepackage[nodisplayskipstretch]{setspace}
\usepackage{hyphenat}
\usepackage{float}
\usepackage{graphicx}
\hyphenation{mate-mática recu-perar}

\titlespacing{\section}{0em}{-0.5em}{0em}
\titlespacing{\subsection}{0pc}{0pt}{0pt}
\titlespacing{\subsubsection}{0pc}{0.33em}{0pc}
\titlespacing{\paragraph}{0em}{0.5em}{0em}
\setlength{\parindent}{2em}
\setlength{\parskip}{1em}
\linespread{1}

\renewcommand{\baselinestretch}{1.0}

\renewcommand\thesubsection{\Alph{subsection}}
\renewcommand\thesection{\Roman{section}}

\renewcommand\bf[1]{\textbf{#1}}
\renewcommand\it[1]{\textit{#1}}

\newcommand\ov[1]{\overline{#1}}
\newcommand{\vect}[1]{\mathbf{#1}}
\newcommand{\vn}{\varnothing}

\newenvironment{nscenter}
 {\parskip=1.25em\par\nopagebreak\centering}
 {\par\noindent\ignorespacesafterend}

\makeatletter
\def\mcolor#1#{\@mcolor{#1}}
\def\@mcolor#1#2#3{%
  \protect\leavevmode
  \begingroup
    \color#1{#2}#3%
  \endgroup
}
\makeatother
\definecolor{notepadrule}{RGB}{217,244,244}

\backgroundsetup{
contents={%
  \begin{tikzpicture}
    \foreach \fila in {0,...,52}
    {
      \draw [line width=1pt,color=notepadrule]
      (current page.west|-0,-\fila*12pt) -- ++(\paperwidth,0);
    }
    \draw[overlay,red!70!black,line width=1pt]
      ([xshift=-1pt]current page text area.west|-current page.north) --
      ([xshift=-1pt]current page text area.west|-current page.south);
  \end{tikzpicture}%
},
scale=1,
angle=0,
opacity=1
}

\begin{document}

\setlength{\abovedisplayskip}{4pt}
\setlength{\belowdisplayskip}{4pt}
\setlength{\abovedisplayshortskip}{0pt}
\setlength{\belowdisplayshortskip}{0pt}
% \setlength{\baselineskip}{12pt}
\setlength{\jot}{2pt}

\section*{Lista 3 - Processos Estocásticos}
Davi de Alencar Mendes (16/0026415) \url{dmendes@aluno.unb.br}

Professor: Dr. Cristiano Jacques Miosso

\section{Problema 1}
Considere que $X: \mathbb{S} \to \mathbb{R}$ é uma variável aleatória com função densidade de
probabilidade $f_X$. Considera ainda a variável aleatória definida pela aplicação da função $g:
\mathbb{R} \to \mathbb{R}$ aos resultados de $X$, com $g(x) = \alpha x + \beta$, sendo $\alpha,
\beta \in \mathbb{R}$ e $\alpha \neq 0$. Assim, $Y = \alpha X + \beta$. A este respeito, responda os
seguintes itens.

\subsection{PDF de Y.}
\begin{align*}
    g(x) &= \alpha x + \beta & g'(x) &= \alpha \\
    g^{-1} &= \frac{y-\beta}{\alpha} = x & f_Y(y) &= \frac{f_X\left(\frac{y-\beta}{\alpha}\right)}{\alpha}
\end{align*}

\subsection{Valor esperado de Y usando valor esperado de X (LOTUS)}
\begin{align*}
    \mathbb{E}[Y] &= \mathbb{E}[g(x)] \\
    \mathbb{E}[Y] &= \int_{-\infty}^{\infty} (\alpha x + \beta)f_X dx = \alpha \mathbb{E}[X] + \beta
\end{align*}

\subsection{Variância de Y usando Variância de X}
\begin{align*}
    Var[X] &= \mathbb{E}[X^2] -(\mathbb{E}[X])^2 \\
    Var[Y] &= \mathbb{E}[Y^2] -(\mathbb{E}[Y])^2 \\
    \mathbb{E}[Y^2] &= \int_{-\infty}^{\infty} (\alpha x + \beta)^2 \cdot f_X dx =
    \int_{-\infty}^{\infty} (\alpha^2 x^2 + + 2 \alpha \beta x + \beta^2) \cdot f_X dx \\
    \mathbb{E}[Y^2] &= \alpha\cdot\int_{-\infty}^{\infty} x^2 f_X dx + 2 \alpha \beta \cdot
    \int_{-\infty}^{\infty} x f_X dx + \beta^2 \cdot \int_{-\infty}^{\infty} f_X dx \\
    \mathbb{E}[Y^2] &= \alpha^2 \mathbb{E}[X^2] + 2 \alpha \beta \mathbb{E}[X] + \beta^2 \\
    Var[Y] &= \alpha^2 \mathbb{E}[X^2] + 2 \alpha \beta \mathbb{E}[X] + \beta^2 - (\alpha^2
    (\mathbb{E}[X])^2 + \beta^2) \\
    Var[Y] &= \alpha^2 Var[X]
\end{align*}

\section{Problema 2}
Considere que $X: \mathbb{S} \to \mathbb{R}$ é uma variável aleatória com distribuição uniforme no
intervalo $[-1,4)$. Seja ainda $Y = 3 X + 8$.

\subsection{PDF de Y}
\begin{align*}
    X &\sim U [-1,4) && Y = 3 X + 8, \alpha=3, \; \beta=8 \\
    f_X(x) &=
    \begin{cases}
        \frac{1}{5}, \text{ se } -1 \leq x < 4, \\
        0 \text{ c.c.} \\
    \end{cases}&&
f_Y(y) = \frac{f_X\left(\frac{y-8}{3}\right)}{3} =
    \begin{cases}
        \frac{1}{15}, \text{ se } 5 \leq y < 20\\
        0 \text{ c.c.}
    \end{cases}
\end{align*}

\subsection{Valor Esperado de Y}
\begin{align*}
    \mathbb{E}[Y] &= \int_{5}^{20} y f_Y(y)dy = \int_{-1}^{4} (3x+8)f_X(x)dx = 3 \mathbb{E}[X] + 8 = 12.5
\end{align*}

\subsection{Variância de Y}
\begin{align*}
    Var[Y] &= \mathbb{E}[Y^2] - (\mathbb{E}[Y])^2 \\
    \ldots\\
    Var[Y] &= \alpha^2 Var[X] = 9 \cdot Var[X] = 18.75
\end{align*}

\section{Problema 3}
\ldots Uma variável X do tipo gaussiano com média 3 e variância 4. Considere $Y = 2X - 6$, e respoda
os itens seguintes.

\subsection{PDF de Y}
\begin{align*}
    X &\sim \mathcal{N}
    \begin{cases}
        \mu = 3\\
        \sigma^2 = 4
    \end{cases}& Y &= 2X - 6 \\
    f_X(x) &= \frac{1}{\sqrt{2\pi\sigma^2}} \cdot \exp{\left(-\frac{(x-\mu)^2}{2\sigma^2}\right)} &
f_y(y) &= \frac{1}{\sqrt{2\pi\sigma^2}} \cdot \exp{\left(-\frac{(\frac{y-\beta}{\alpha}-\mu)^2}{2\sigma^2}\right)}
\end{align*}

\subsection{Média e Variância de $Y$}
\begin{align*}
    \mathbb{E}[Y] &= \alpha \mathbb{E}[X] + \beta = 2\cdot3 - 6 = 0 \\
    Var[Y] &= \alpha^2 Var[X] = 2^2 \cdot 4 = 16
\end{align*}

\subsection{Calcule o momento central de ordem 3 normalizado. O valor era esperado?}
Pela definição (para uma variável aleatória $X$):
\begin{align*}
    Skew[X] &= \widetilde{\mu}_3 = \mathbb{E}\left[ \left( \frac{X - \mu}{\sigma} \right)^3
    \right] = \frac{\mu_3}{\sigma^3} \\
    Skew[X] &= \frac{\mathbb{E}[X^3] -3\mu\mathbb{E}[X^2] +3\mu^2\mathbb{E}[X] -
    \mu^3}{\sigma^3} \\
    Skew[X] &= \frac{\mathbb{E}[X^3] -3\mu(\mathbb{E}[X^2] -\mu\mathbb{E}[X])-\mu^3}{\sigma^3} \\
    Skew[X] &= \frac{\mathbb{E}[X^3] -3\mu\sigma^2 -\mu^3}{\sigma^3}
\end{align*}

Buscando $\mathbb{E}[Y^3]$:
\begin{align*}
    \mathbb{E}[Y^3] &= \int_{-\infty}^{\infty} y^3 f_Y(y)dy = \int_{-\infty}^{\infty} (2x-6)^3f_X(x)dx \\
    \mathbb{E}[Y^3] &= \int_{-\infty}^{\infty} (8x^3-72x^2+216x-216)f_X(x)dx \\
    \mathbb{E}[Y^3] &= \int_{-\infty}^{\infty} (8x^3)f_X(x)dx +
                       \int_{-\infty}^{\infty} (-72x^2)f_X(x)dx +
                       \int_{-\infty}^{\infty} (216x)f_X(x)dx +
                       \int_{-\infty}^{\infty} (-216)f_X(x)dx \\
    \mathbb{E}[Y^3] &= \int_{-\infty}^{\infty} (8x^3)f_X(x)dx +
                       \int_{-\infty}^{\infty} (-72x^2)f_X(x)dx +
                       216\cdot\mathbb{E}[X] + 216
\end{align*}
    % \mathbb{E}[Y^3] &=
As integrais restantes são integrais de Laplace com solução para:
\begin{align*}
    \int_{-\infty}^{\infty} z^n \cdot e^{-\alpha \cdot z^n} &= \frac{1\cdot3\cdot5\cdot\ldots(n+1)\pi^{1/2}}{2^{n/2}\alpha^{(n+1)/2}}
\end{align*}

Aplicando para as integrais anteriores a mudança: $z = x - \mu, dx = dz$ é possível obter
(\textit{omitindo boa parte do desenvolvimento})
\begin{align*}
    \mathbb{E}[Y^3] &= 8\cdot(\mu\cdot\sigma^2 + \mu^3) -72\cdot(\mu^2 + \sigma^2) + 216\cdot\mu + 216 \\
    \mathbb{E}[Y^3] &= 8\cdot(3\cdot3\cdot4+3^3) -72\cdot(3^3+4) + 216\cdot3 + 216 = 0 \\
    Skew[Y] &= 0
\end{align*}

O valor era esperado já que VAs do tipo gaussiano são totalmente descritas em termos de média e
variância (2 momentos somente).

% TODO gráfico Q3
\subsection{Trace o gráfico da PDF de $Y$ e compare com o de $X$. O que muda?}

\section{Problema 4}
Seja $X$ um vetor aleatório de 2 componentes com distribuição uniforme em $[-1,4]\times[-2,8]$.
Responda os seguintes itens.

\subsection{PDF Conjunta de $X$}
\begin{align*}
    X &= \begin{bmatrix} x_1\\x_2 \end{bmatrix} & X &\sim \mathcal{U}[-1,4]\times[-2,8]
\end{align*}%
\begin{align*}
    \int_{{-1}}^{{4}} {\int_{-2}^{8}c \cdot dx_1} \: d{x_2} = 1, \quad c = \frac{1}{50} \qquad
    f_X(x_1,x_2) = \frac{1}{50}
\end{align*}

\subsection{PDF Marginal da 1a componente}
\begin{align*}
    f_{x_1}(x_1) &=\int_{-2}^{8} \frac{1}{50}dx_2 = \frac{1}{5}
\end{align*}

\subsection{PDF Marginal da 2a componente}
\begin{align*}
    f_{x_2}(x_2) &=\int_{-1}^{4} \frac{1}{50}dx_1 = \frac{1}{10}
\end{align*}

\subsection{Vetor de valores esperados}
\begin{align*}
    \mathbb{E}[X] = \begin{bmatrix} 1.5\\3 \end{bmatrix}
\end{align*}

\subsection{Vetor de variâncias}
\begin{align*}
    \mathbb{E}[x_1^2] &= \int_{-1}^4 x_1^2 \frac{1}{5}dx_1 = \frac{x^3}{15} \bigg\rvert_{-1}^{4} = \frac{13}{3} \\
    Var[x_1] &= \frac{13}{3} - (1.5)^2 = \frac{25}{12} \\
    \mathbb{E}[x_2^2] &= \int_{-2}^8 x_2^2 \frac{1}{10}dx_2 = \frac{x^3}{10} \bigg\rvert_{-2}^{8} = \frac{52}{3} \\
    Var[x_2] &= \frac{52}{3} - (3)^2 = \frac{25}{3} \\
    Var[X] &= \begin{bmatrix} \frac{25}{12} & \frac{25}{3} \end{bmatrix}^T
\end{align*}

\subsection{Matriz de Autocovariâncias}
\begin{align*}
    R_X &= \mathbb{E}[XX^T] \\
    C_X &= \mathbb{E}[(X-\mu_X)(X-\mu_X)^T] \\
    C_X &= R_X - \mu_X\mu_X^T \\
    R_X &= \begin{bmatrix} \mathbb{E}[x_1^2] & \mathbb{E}[x_1x_2]\\\mathbb{E}[x_2x_1] & \mathbb{E}[x_2^2] \end{bmatrix}  \\
    \mathbb{E}[x_2x_1] &= \mathbb{E}[x_1x_2] = \int_{-1}^4 \int_{-2}^8 x_1x_2 \cdot \frac{1}{50}
    dx_1 dx_2 = \frac{9}{2} \\
\end{align*}
\begin{align*}
    R_X &= \begin{bmatrix} \frac{13}{3} & \frac{9}{2}\\\frac{9}{2} & \frac{52}{3} \end{bmatrix} &
    \mu_X\mu_X^T &= \begin{bmatrix} 2.25 & 4.5\\4.5 & 9 \end{bmatrix} \\
    C_X &= \begin{bmatrix} \frac{25}{12} & 0\\0 & \frac{26}{3} \end{bmatrix}
\end{align*}

\subsection{As componentes são correlatas ou não?}
Não são! $C_X$ é matrix diagonal.

\subsection{As componentes são dependentes ou não?}
São independentes visto que a PDF pode ser fatorada.

\section{Problema 5}
Sobre o Problema 4, determine:

\subsection{$P(X_2 > X_1)$?}
\begin{align*}
    P(X_2 > X_1) &= \int_{-1}^4 \int_{x_1}^8 \frac{1}{50} dx_2 dx_1 = 0.65  \\
    P(X_1 > X_2) &= \int_{-1}^4 \int_{-2}^{x_1} \frac{1}{50} dx_2 dx_1 = 0.35
\end{align*}

\subsection{$P(X_2 > 2X_1)$?}
\begin{align*}
    P(X_2 > 2X_1) &= \int_{-1}^4 \int_{2x_1}^8 \frac{1}{50} dx_2 dx_1 = 0.5
\end{align*}

\section{Problema 6}
\begin{align*}
    f_X(x_1,x_2,x_3) &= \frac{3}{4\pi a^3}, \text{ se } \sqrt{x_1+x_2+x_3} < a \\
    P[A] &= \int\int\int_S f_X(x_1,x_2,x_3) dx_1 dx_2 dx_3, \text{ com } S = \{\sqrt{x_1+x_2+x_3} < 2a/3\} \\
    \text{ Em coordenadas esféricas}&
    \begin{cases}
        r^2 = x_1^2 + x_2^2 + x_3^2 \\
        x_1 = r \sin\phi \cos\theta \\
        x_2 = r \sin\phi \sin\theta \\
        x_3 = r \cos\phi
    \end{cases} dx_1 dx_2 dx_3 = r^2 \sin\phi dr d\phi d\theta \\
    P[A] &= \frac{3}{4\pi a^3} \int_{0}^{2a/3} \int_{0}^{\pi} \int_{0}^{2\pi} r^2 \sin\phi \; d\theta d\phi dr
\end{align*}

A integral é bastante simples e corresponde a uma fração do volume total já que $f_X$ é uniforme ao
longo do raio. Por esse motivo $P[A]$ pode ser obtido como uma razão dos volumes: $(2a/3)/a^3 = 8/27 \approx 0.3$.

\section{Problema 7}
Considere que $X$ é um vetor aleatório de 2 componentes com:
\begin{align*}
    C_{XX} &= \begin{bmatrix} 4 & 1\\1 & 4 \end{bmatrix} \qquad \mu_X = \begin{bmatrix} 2\\3 \end{bmatrix} \\
    Y &= AX + b, \qquad A = \begin{bmatrix} 1 & 2\\-2 & 3 \end{bmatrix}, \;\; b = \begin{bmatrix} 1\\4 \end{bmatrix}
\end{align*}

\subsection{Matriz de Autocovariâncias de $Y$?}
\begin{align*}
    \mu_Y &= \mathbb{E}[AX + b] = A\mu_X + b \\
    Y - \mu_Y &= A(X -\mu_X) \\
    C_{YY} &= \mathbb{E}[(Y-\mu_Y)(Y-\mu_Y)^H] \\
    C_{YY} &= \mathbb{E}[(A(X-\mu_X))(A(X-\mu_X))^H] \\
    \text{com } (AB)^T &= B^TA^T \text{ e também } (AB)^H = B^HA^H \\
    (A(X-\mu_X))^H &= (X-\mu_X)^HA^H \\
    C_{YY} &= \mathbb{E}[A(X-\mu_X)(X-\mu_X)^HA^H] \\
    C_{YY} &= A\mathbb{E}[(X-\mu_X)(X-\mu_X)^H]A^H \\
    C_{YY} &= A C_{XX} A^H
\end{align*}

Aplicando em $C_{XX}$:
\begin{align*}
    C_{YY} = \begin{bmatrix} 1&2\\-2&3 \end{bmatrix} \cdot
    \begin{bmatrix} 4&1\\1&4 \end{bmatrix} \cdot
    \begin{bmatrix} 1&-2\\2&3 \end{bmatrix} =
    \begin{bmatrix} 24&15\\15&40 \end{bmatrix}
\end{align*}

\subsection{Matriz de autocorrelações de $Y$?}
\begin{align*}
    R_{YY} &= \mathbb{E}[YY^H] \\
    R_{YY} &= \mathbb{E}[(AX+b)(AX+b)^H] \\
    R_{YY} &= \mathbb{E}[(AX+b)((AX)^H+b^H)] \\
    R_{YY} &= \mathbb{E}[AXX^HA^H + AXb^H + bX^HA^H + bb^H] \\
    R_{YY} &= AR_{XX}A^H + (A\mu_X)b^H + b(A\mu_X)^H + bb^H
\end{align*}

Buscando $R_{XX}$:
\begin{align*}
    C_{XX} &= R_{XX} - \mu_X\mu_X^T \\
    R_{XX} &= \begin{bmatrix} 8&7\\7&13 \end{bmatrix} \\
    R_{YY} &= \begin{bmatrix} 105&96\\96&121 \end{bmatrix}
\end{align*}

\subsection{Valor Médio de $Y$}
\begin{align*}
    \mu_Y &= A \cdot \mu_X + b = \begin{bmatrix} 1 & 2\\-2 & 3 \end{bmatrix} \cdot \begin{bmatrix}
    2\\3 \end{bmatrix}  + \begin{bmatrix} 1\\4 \end{bmatrix} = \begin{bmatrix} 9\\9 \end{bmatrix}
\end{align*}

\subsection{O vetor $Y$ é correlato?}
Sim, o vetor é correlato. $C_{YY}$ não é matriz diagonal.

\subsection{O vetor $Y$ é independente?}
Não é independente.

\subsection{Qual a PDF de Y, supondo distribuição gaussiana?}
\begin{align*}
    f_Y(\mathbf{y}) &= \frac{1}{\sqrt{(2\pi)^2 \det{C_{YY}}}} \cdot \exp{\left(-\frac{1}{2}(y-\mu_Y)^T C_{YY}^{-1} (y-\mu_Y)\right)} \\
    C_{YY}^{-1} &= \begin{bmatrix} \frac{8}{147} & \frac{-1}{49}\\\frac{-1}{49} & \frac{8}{245} \end{bmatrix}, \qquad \det{C_{YY}} = 735, \\
    f_Y(\mathbf{y}) &= \frac{1}{\sqrt{(2\pi)^2 \cdot 735}} \cdot \exp{\left(-\frac{4y_1^2}{147} + y_1\frac{(y_2+15)}{49} -\frac{4y_2^2}{245} + \frac{27y_2}{245} -\frac{459}{245}\right)}\\
\end{align*}

\subsection{Matriz de Transformação $M$ e  vetor $v$ tal que $Z = MY + v$ seja não correlato e
possua média nula}

Para obter o vetor $v$ basta:
\begin{align*}
    \mathbb{E}[Z] &= \mathbb{E}[MY + v] = M\mu_Y + v = 0 \\
    v &= M\mu_Y
\end{align*}

\subsubsection{EVD - Decomposição Espectral (\textit{Eigenvalue Decomposition})}
Em uma matriz $A_{N \times N}$, Solucionando $(A -\lambda I)k = 0$ é possível encontrar os
autovetores associados aos autovalores $k$. O polinômio característico $p(\lambda) = det(A - \lambda
I)$ possui N raízes, algumas com possível multiplicidade maior que 1.

Pelas características dos autovetores é possível decompor:
\begin{align*}
    Av &= \lambda v \\
    A Q &= Q \Lambda \\
    A &= Q \Lambda Q^{-1}
\end{align*}
\begin{itemize}
    \item $A_{N \times N}$ - matriz com $n$ autovetores $q_i$ linearmente independentes
    \item $\Lambda$ - matriz diagonal com autovalores ao longo da diagonal principal ($\Lambda_{ii} = \lambda_i$)
    \item $Q$ - matriz com i-ésima coluna sendo autovetor $q_i$ de $A$.
\end{itemize}

Algumas propriedades:
\begin{enumerate}
    \item Se $A$ é simétrica, $Q$ (matriz formada pelos autovetores de $A$) é ortogonal com $Q^{-1} = Q^T$
    \item $[\Lambda^{-1}]_{ii} = 1/\lambda_i$, pois $\Lambda$ é diagonal
    \item Se $A$ é hermitiana e positiva definida, $Q$ será unitária com $Q^{-1} = Q^H$
\end{enumerate}

\subsubsection{Buscando a Matriz de Transformação $M$}
\begin{align*}
    C_{ZZ} &= M C_{YY} M^H, \text{ fazendo } C_{YY} = Q \Lambda Q^{-1} \\
    C_{ZZ} &= M Q \Lambda Q^{-1} M^H = M Q \Lambda Q^H M^H, \text{ fazendo } M = Q^H = Q^{-1} \\
    C_{ZZ} &= Q^{-1}Q \Lambda Q^{-1} Q = I \Lambda I = \Lambda \\
    C_{ZZ} &= \Lambda, \text{ para } M = Q^H
\end{align*}

Os dados do problema são:
\begin{align*}
    C_{YY} &= \begin{bmatrix} 24 & 15\\15 & 40 \end{bmatrix} \qquad p(\lambda) = \lambda^2 -64\lambda + 735
    \begin{cases}
        \lambda_1 = 15\\
        \lambda_2 = 49
    \end{cases}\\
    C_{YY} - 15I &= \begin{bmatrix} 9 & 15\\15 & 25 \end{bmatrix} \qquad \text{ (RREF) }
    \begin{bmatrix} 1 & 5/3\\0 & 0 \end{bmatrix} \begin{bmatrix} m_1\\m_2 \end{bmatrix} = m_1 +\frac{5}{3}m_2 = 0 \\
    v_1 &= \begin{bmatrix} 1 \\ 5/3 \end{bmatrix}, \qquad \widehat{v_1} = \frac{\sqrt{34}}{34} \begin{bmatrix} 5\\-3  \end{bmatrix} \\
    C_{YY} - 49I &= \begin{bmatrix} -25 & 15\\15 & -9 \end{bmatrix} \qquad \text{ (RREF) }
    \begin{bmatrix} 1 & -3/5\\0 & 0 \end{bmatrix} \begin{bmatrix} m_1\\m_2 \end{bmatrix} = m_1 -\frac{3}{5}m_2 = 0 \\
    v_2 &= \begin{bmatrix} 1 \\ -3/5 \end{bmatrix}, \qquad \widehat{v_2} = \frac{\sqrt{34}}{34} \begin{bmatrix} 3\\5  \end{bmatrix} \\
\end{align*}

Finalmente, obtemos $Q, \Lambda, v$:
\begin{align*}
    Q &= \frac{\sqrt{34}}{34} \begin{bmatrix} 5&3\\-3&5 \end{bmatrix} \\
    \Lambda &= \begin{bmatrix} 15&0\\0&49 \end{bmatrix} \\
    M &= Q^{-1} = \frac{\sqrt{34}}{34} \begin{bmatrix} 5&-3\\3&5 \end{bmatrix} \\
v &= M\mu_Y = \frac{\sqrt{34}}{17} \begin{bmatrix} 9\\36 \end{bmatrix}
\end{align*}

\section{Problema 8}
Considere que $X$ é um vetor aleatório com função de densidade de probabilidade dada por
\begin{align*}
    f_X(x_1, x_2) &= \frac{\sqrt{2}}{16\pi} \exp{\left(-\frac{9}{64}x_1^2+\frac{1}{16}x_1x_2-\frac{1}{16}x_2^2\right)}
\end{align*}

\subsection{Calcule $M$ tal que $Y = MX$ seja um vetor não correlato}
É possível usar uma PDF gaussiana multivariada como protótipo para $f_X(\mathbf{x})$ de maneira que
\begin{align*}
    f_X(\mathbf{x}) &= \frac{1}{\sqrt{(2\pi)^2 \det{C_{XX}}}} \cdot \exp{\left(-\frac{1}{2}\mathbf{x}^T C_{XX}^{-1} \mathbf{x}\right)} \\
C_{XX}^{-1} &= \begin{bmatrix} a&b\\c&d \end{bmatrix} \qquad
C_{XX} = \frac{1}{ad-bc} \begin{bmatrix} d&-b\\-c&a \end{bmatrix} \qquad \det{C_{XX}} = (4\cdot\sqrt{2})^2 \\
\left(-\frac{1}{2}\mathbf{x}^T C_{XX}^{-1} \mathbf{x}\right) &= \left(-\frac{9}{64}x_1^2+\frac{1}{16}x_1x_2-\frac{1}{16}x_2^2\right)
\end{align*}

Buscando $C_{XX}^{-1}$
\begin{align*}
\left(-\frac{1}{2}\mathbf{x}^T C_{XX}^{-1} \mathbf{x}\right) &= \left(-\frac{9}{64}x_1^2+\frac{1}{16}x_1x_2-\frac{1}{16}x_2^2\right)\\
-\frac{1}{2}\left(ax_1^2 + (b+c)x_1x_2 + dx_2^2\right) &= \left(-\frac{9}{64}x_1^2+\frac{1}{16}x_1x_2-\frac{1}{16}x_2^2\right) \\
a = 9/32 \quad b = c  &= -1/16  \quad d = 1/8, \text{ b é igual a c por simetria de } C_{XX}
\end{align*}

Montando $C_{XX}$:
\begin{align*}
    C_{XX}^{-1} &= \begin{bmatrix} 9/32 & -1/16 \\ -1/16 & 1/8 \end{bmatrix} \quad C_{XX} = \begin{bmatrix} 4 & 2 \\ 2 & 9 \end{bmatrix}
\end{align*}

Aplicando a KLT com a matriz $M$ e realizando EVD de $C_{XX}$
\begin{align*}
C_{YY} &= M C_{XX} M^H \\
C_{XX} &= Q \Lambda Q^{-1} \\
\text{com } M &= Q^{-1}, \; C_{YY} = \Lambda
\end{align*}

\begin{align*}
    Q &= \begin{bmatrix} 1 & 1 \\ -\frac{(\sqrt{41}-5)}{4} & \frac{\sqrt{41}+5}{4} \end{bmatrix} \qquad
    \Lambda = \begin{bmatrix} -\frac{(\sqrt{41}-13)}{2} & 0 \\ 0 & \frac{(\sqrt{41}+13)}{2}\end{bmatrix} \qquad
    M = Q^{-1} = \begin{bmatrix} \frac{5\sqrt{41}}{82}+\frac{1}{2} & -\frac{2\sqrt{41}}{41} \\ \frac{1}{2}\frac{-5\sqrt{41}}{82} & \frac{2\sqrt{41}}{41}\end{bmatrix}
\end{align*}

\subsection{O vetor $Y$ é independente ou dependente?}
$C_{YY} = \Lambda, \det{C_{YY}} \neq 0$. É independente (pdf pode ser fatorada) em duas
exponenciais.

% TODO
\subsection{Forneça gráfico da PDF de $X$}
\subsection{Forneça gráfico da PDF de $Y$ e compare como o item anterior}

\section{Problema 9}
Seja $X: \mathbb{S} \to \mathbb{R}^N$ um vetor aleatório, com $N$ sendo inteiro positivo. Seja ainda
$X_i$ a variável aleatória correspondente à i-ésima componente de $X$, com $i \in \{1,2,\ldots,N\}$.
Responda os itens.

\subsection{Determine a PDF de $X_1 + X_2$. Qual a operação entre as PDFs de $X_1$ e $X_2$ que gera
essa PDF?}
Iniciamos com $g(X) = Y = X_1 + X_2$. Para tal
\begin{align*}
    h(x_1,x_2) &= \begin{bmatrix} x_1+x_2\\x_1 \end{bmatrix}
    \qquad J_h = \det{\left(\frac{\partial h}{\partial x}\right)} = \begin{vmatrix} 1&1\\1&0 \end{vmatrix} = 1
\end{align*}

Com $w \in \mathbb{R}^2$, temos a PDF de $W$ no ponto $w$:
\begin{align*}
f_W(w) = \sum_i \frac{f_X(x^{(i)})}{|\det{J_h(x^{(i)})}}, \text{ com } \{x^{(1)},x^{(2)},\ldots\}
\end{align*}
Conjunto de todos os vetores $x^{(i)}$ tais que $g(x^{(i)}) = w \ \forall \; w \in \mathbb{R}^2,
x^{(i)} = \begin{bmatrix} w_2 \\ w_1-w_2 \end{bmatrix} $

Fazemos:
\begin{align*}
    f_Y(y) &= \int_{-\infty}^{\infty} f_x(w_2, y - w_2) d{w_2} \text{ (caso geral)} \\
    &\text{Para o caso em que as componentes de X são independentes:} \\
    f_Y(y) &= \int_{-\infty}^{\infty} f_X(w_2) \cdot f_X(y-w_2) d{w_2} \text{ (note que trata-se de uma convolução!)}
\end{align*}

\subsection{Determine PDF de $X_1^2 + X_2^2$ para $X_1, X_2$ gaussianas e independentes entre si}

\subsection{As variáveis $X_1, X_2$ são independentes?}
Considere que a correlação entre $X_1$ e $X_2$ vale 400, que as médias de $X_1$ e $X_2$ são,
respectivamente, 10 e 20, que as variância sejam 4 e 16 respectivamente.

Pelos dados da questão:
\begin{align*}
    \mathbb{E}[X_2X_1] = \mathbb{E}[X_1X_2] = 400 \quad \mu_X = \begin{bmatrix} 10\\20 \end{bmatrix} \quad Var[X] = \begin{bmatrix} 4\\16 \end{bmatrix}
\end{align*}

Buscando a matriz de correlação
\begin{align*}
    Var[X_i] &= \mathbb{E}[X_i^2] - (\mathbb{E}[X_i])^2 \\
    \begin{bmatrix} 4\\16 \end{bmatrix} &= \mathbb{E}[X^2] - \left(\begin{bmatrix} 10\\20 \end{bmatrix}\right)^2 \\
    \mathbb{E}[X^2] &= \begin{bmatrix} 104\\416 \end{bmatrix} \\
    R_{XX} &=  \begin{bmatrix} \mathbb{E}[X_1^2]&\mathbb{E}[X_1X_2] \\ \mathbb{E}[X_2X_1]&\mathbb{E}[X_2^2] \end{bmatrix} = \begin{bmatrix} 104&400\\400&416 \end{bmatrix} \\
    C_{XX} &= R_{XX} -\mu_X\mu_X^T = \begin{bmatrix} 4&200\\200&16 \end{bmatrix}
\end{align*}

SOS: Algo deu errado aqui!  $C_{XX}$ possui autovalores negativos. Determinante igualmente negtivo!

\subsection{Igual ao anterior (?)}
O problema D está igual ao item C. Possível \textit{typo}.

% \subsection{Determine PDF Conjunta usando item os dados do item anterior}
% Bem, basta aplicar:
% \begin{align*}
%     f_X(\mathbf{x}) &= \frac{1}{\sqrt{(2\pi)^2 \det{C_{XX}}}} \cdot \exp{\left(-\frac{1}{2}(x-\mu_X)^T C_{XX}^{-1} (x-\mu_X)\right)} \\
%     \left(-\frac{1}{2}(x-\mu_X)^T C_{XX}^{-1} (x-\mu_X)\right) &= \left(\frac{x_2^1}{4992}+x_1\cdot\left(\frac{5}{52}-\frac{25x_2}{4992}\right)+\frac{x_2^2}{19968}+\frac{5x_2}{105}-\frac{5}{26}\right)
% \end{align*}

\section{Problema 10}
Seja $X: \mathbb{S} \to \mathbb{R}^N$ um vetor aleatório, com distribuição gaussiana.

Repetindo parte do procedimento do exercício anterior:
\begin{align*}
    Var[X] &= \begin{bmatrix} 1\\4 \end{bmatrix} \quad \mu_X = \begin{bmatrix} 1\\4 \end{bmatrix}
    \quad \mathbb{E}[X^2] = \begin{bmatrix} 2\\20 \end{bmatrix} \\
    R_{XX} &= \begin{bmatrix} 2&100\\100&20 \end{bmatrix} \quad C_{XX} = \begin{bmatrix} 1&96\\96&4 \end{bmatrix}
\end{align*}

Algo está errado (???)
$C_{XX}$ possui autovalores negativos. Determinante igualmente negtivo!

\textit{That's all Folks}
\end{document}
