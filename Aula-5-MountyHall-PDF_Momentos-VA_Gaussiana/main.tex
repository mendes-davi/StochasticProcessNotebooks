\documentclass{article}
\usepackage{enumitem}
\usepackage[T1]{fontenc}
\usepackage[utf8]{inputenc}
\usepackage[portuguese]{babel}
\usepackage[vmargin=3cm]{geometry}
\usepackage{tikzpagenodes}
\usepackage{lipsum}
\usepackage{xcolor}
\usepackage{cancel}
\usepackage{amsmath}
\usepackage{amssymb}
\usepackage{background}
\usepackage{titlesec}
\usepackage[nodisplayskipstretch]{setspace}
\usepackage{hyphenat}
\usepackage[normalem]{ulem}
\hyphenation{mate-mática recu-perar}

\titlespacing{\section}{0pc}{-0.25em}{0pc}
\titlespacing{\subsection}{0pc}{0em}{0pc}
\titlespacing{\subsubsection}{0pc}{0.33em}{0pc}
\titlespacing{\paragraph}{0em}{0.125em}{0.5em}
\setlength{\parindent}{2em}
\setlength{\parskip}{1em}
\linespread{1}

\renewcommand{\baselinestretch}{1.0}

\renewcommand\bf[1]{\textbf{#1}}
\renewcommand\it[1]{\textit{#1}}

\newcommand\ov[1]{\overline{#1}}
\newcommand{\vect}[1]{\mathbf{#1}}
\newcommand{\vn}{\varnothing}
\newcommand\stk[2][black]{\setbox0=\hbox{$#2$}%
\rlap{\raisebox{.45\ht0}{\textcolor{#1}{\rule{\wd0}{1pt}}}}#2}

\makeatletter
\def\mcolor#1#{\@mcolor{#1}}
\def\@mcolor#1#2#3{%
  \protect\leavevmode
  \begingroup
    \color#1{#2}#3%
  \endgroup
}
\makeatother
\definecolor{notepadrule}{RGB}{217,244,244}

\backgroundsetup{
contents={%
  \begin{tikzpicture}
    \foreach \fila in {0,...,52}
    {
      \draw [line width=1pt,color=notepadrule]
      (current page.west|-0,-\fila*12pt) -- ++(\paperwidth,0);
    }
    \draw[overlay,red!70!black,line width=1pt]
      ([xshift=-1pt]current page text area.west|-current page.north) --
      ([xshift=-1pt]current page text area.west|-current page.south);
  \end{tikzpicture}%
},
scale=1,
angle=0,
opacity=1
}

\begin{document}

\setlength{\abovedisplayskip}{12pt}
\setlength{\belowdisplayskip}{12pt}
\setlength{\abovedisplayshortskip}{0pt}
\setlength{\belowdisplayshortskip}{0pt}
% \setlength{\baselineskip}{12pt}
\setlength{\jot}{0pt}

\section{Um breve exercício: \it{Mounty Hall Problem}}
Suponha 3 portas que escondem um objeto. Atrás de uma porta, há a representação de um carro.
Ademais, cada uma das outras portas esconde o desenho de uma cabra.

Um apresentador solicita que um participante escolha uma das portas. A ideia é que se o
participante acerta a (porta) que está a frente do carro, ele leva o carro como prêmio!! O
apresentador sempre mostra uma das portas após a escolha inicial, revelando uma cabra. Ele então
questiona o participante se ele deseja trocar sua escolha.
\\
\bf{Pergunta:} Para tentar aumentar a chance de ganhar, o participante deve trocar, deve manter ou
não faz diferença?

\section{Proprieades da PDF (continuação)}
\paragraph{Momentos}
Para a variável aleatória $X$: são números que traduzem informação sobre o comportamento
estatístico de uma variável, em crescente nível de detalhamento acumulado.

\bf{Definição formal:} Momentos de orden $m$ (momento não-central de ordem $m$) $M_m =
\int^\infty_{-\infty}
x^mf_X(x)dx$, $M_1 = \mathbb{E}(X)$, $M_0 = 1 \; \forall $ v.a.
\\[0.25em]
O momento central em torno de $\mathbb{E}(X)$ é $C_m = \int^\infty_{-\infty} (x-\mathbb{E}(X))^m f_Xdx$
\\
O momento central em torno de $k$ é $C_m = \int^\infty_{-\infty} (x-k)^m f_Xdx$

Na prática, conseguimos estimar momentos (usando estimadores). Ter um estimador preciso é tão mais
difícil quanto maior for a ordem. Nem toda v.a. possui todos os momentos (a integral pode não
convergir).

Quandos os momentos existem, conhecer uma PDF equivale a conhecer infinitos momentos. Ou seja,
saber N momentos não nos permite conhecer qual a PDF.

\section{Variável Aleatória Gaussiana}
Uma variável aletória $X$ é dita normal ou Gaussiana quando sua PDF é data por:

\begin{align*}
    f_X(x) &= \frac{1}{\sqrt{2\pi A}} \exp{[-\frac{1}{2} \frac{(x-B)^2}{A}]}, \; \forall \; x, A, B
    \in \mathbb{R}, A > 0. \\
    \text{alternativamente: } f_X(x) &= \frac{1}{\sqrt{2\pi \sigma^2}} \exp{[-\frac{1}{2} \frac{(x-\mu)^2}{\mu}]} \\
    \\
           &\text{Demonstrar:} \\
    \\
    f_X(x) &> 0, \; \forall \; x \in \mathbb{R} \\
    \int^\infty_{-\infty} f_X(x) &= 1 \it{ confira demonstração de Poisson}
\end{align*}
\\
\bf{Obs:} $\mathbb{E}(X) = \int^\infty_{-\infty} x \cdot f_X(x)dx = B = M_1$ e
$Var(X) = A = C_2$

Observe que ao se tratar de PDF Gaussiana os 2 momentos servem para descrever a v.a. em sua
totalidade. O ponto de máximo de uma PDF é o valor de $x$ denominado moda. Na Gaussiana: moda,
média e mediana são iguais. Mediana é o x tal que: $\int^x_{-\infty} f_X(\lambda)d\lambda = \int^\infty_{x} f_X(\lambda) d\lambda = 0.5$

\end{document}
