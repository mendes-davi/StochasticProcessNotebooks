\documentclass{article}
\usepackage{enumitem}
\usepackage[T1]{fontenc}
\usepackage[utf8]{inputenc}
\usepackage[portuguese]{babel}
\usepackage[vmargin=3cm]{geometry}
\usepackage{tikzpagenodes}
\usepackage{lipsum}
\usepackage{xcolor}
\usepackage{cancel}
\usepackage{amsmath}
\usepackage{mathrsfs}
\usepackage{amssymb}
\usepackage{background}
\usepackage{titlesec}
\usepackage[nodisplayskipstretch]{setspace}
\usepackage{hyphenat}
\usepackage[normalem]{ulem}
\usepackage{subcaption}
\hyphenation{mate-mática recu-perar}

\titlespacing{\section}{0pc}{-0.25em}{0pc}
\titlespacing{\subsection}{0pc}{0em}{0pc}
\titlespacing{\subsubsection}{0pc}{0.33em}{0pc}
\titlespacing{\paragraph}{0em}{0.25em}{0.5em}
\setlength{\parindent}{2em}
\setlength{\parskip}{1em}
\linespread{1}

\renewcommand{\baselinestretch}{1.0}

\renewcommand\bf[1]{\textbf{#1}}
\renewcommand\it[1]{\textit{#1}}

\newcommand\ov[1]{\overline{#1}}
\renewcommand\u[1]{\underbar{#1}}
\newcommand{\vect}[1]{\mathbf{#1}}
\newcommand{\bb}[1]{\mathbb{#1}}
\newcommand{\vn}{\varnothing}
\newcommand\stk[2][black]{\setbox0=\hbox{$#2$}%
\rlap{\raisebox{.45\ht0}{\textcolor{#1}{\rule{\wd0}{1pt}}}}#2}

\makeatletter
\global\let\tikz@ensure@dollar@catcode=\relax
\makeatother

\makeatletter
\def\mcolor#1#{\@mcolor{#1}}
\def\@mcolor#1#2#3{%
  \protect\leavevmode
  \begingroup
    \color#1{#2}#3%
  \endgroup
}
\makeatother
\definecolor{notepadrule}{RGB}{217,244,244}

\backgroundsetup{
contents={%
  \begin{tikzpicture}
    \foreach \fila in {0,...,52}
    {
      \draw [line width=1pt,color=notepadrule]
      (current page.west|-0,-\fila*12pt) -- ++(\paperwidth,0);
    }
    \draw[overlay,red!70!black,line width=1pt]
      ([xshift=-1pt]current page text area.west|-current page.north) --
      ([xshift=-1pt]current page text area.west|-current page.south);
  \end{tikzpicture}%
},
scale=1,
angle=0,
opacity=1
}

\begin{document}

\setlength{\abovedisplayskip}{12pt}
\setlength{\belowdisplayskip}{12pt}
\setlength{\abovedisplayshortskip}{0pt}
\setlength{\belowdisplayshortskip}{0pt}
% \setlength{\baselineskip}{12pt}
\setlength{\jot}{1pt}

\section{Ergodicidade}
Um processo é dito ergódico com respeito a um dado momento quando um estimador desse momento quando
um estimador desse momento quando aplicado a diferentes intervalos de uma realização do sinal tem a
mesma expectância que o estimador aplicado a um único intervalo em realizações distintas do
processo.

Tomamos um sinal ergódico na média (\textit{mean-ergodic}). Em uma realização, temos um estimador
dentro de um sinal $\overline{\mu} = \frac{1}{N} \sum_{n=0}^{N-1} x[n]$. Para diversas realizações
(método pela definição) temos $\overline{\mu[2]} = \frac{1}{M} \sum_{m=0}^{M-1} x_m[2]$. No caso da
ergodicidade para a média esses valores se igual e tendem a zero quando a quantidade de realizações
cresce indefinidamente.

\vspace{1em}
\section{Teorema de Wiener-Khinchin-Einstein}
Considere um processo estocástico $X_c: S \to E_S$ do tipo WSS. Considere ainda $g: E_S \to E_S$,
com $E_S$ um espaço de sinais que admitem transformada de Fourier e com $g(x) = \mathcal{F}\{x\}$.

Note que $g(x)$ induz um novo processo $\widehat{X}: S \to E_S$, definido por: $\widehat{X}(s) =
g(X(s))$.

Nesta situação:
\begin{align*}
    \widehat{r}(f) = \lim_{T \to \infty} \left[ \frac{1}{T} \mathbb{E}[|\widehat{X_f}|^2 \cdot
rect_{[-\frac{T}{2}, \frac{T}{2}]}(f)] \right] \end{align*}

\vspace{0.25em}
No caso de um processo branco: por definição $\frac{1}{T} \mathbb{E}[|\widehat{X_f}|^2 \cdot
rect_{[-\frac{T}{2}, \frac{T}{2}]}(f)]$ é constante, ou seja, independente de f.

Pelo teorema: $\frac{1}{T} \mathbb{E}[|\widehat{X_f}|^2 \cdot
rect_{[-\frac{T}{2}, \frac{T}{2}]}(f)]$ tem que coincidir com a transformada de Fourier da
autocorrelação logo, a T.F. de $r(t)$ tem que ser constante. Dessa maneira, $r(t) = k \delta(t)$.

\textbf{Sintetizando o teorema:} A T.F da autocorrelação é a DEP (PSD), para um processo WSS.


\end{document}
