\documentclass{article}
\usepackage{enumitem}
\usepackage[T1]{fontenc}
\usepackage[utf8]{inputenc}
\usepackage[portuguese]{babel}
\usepackage[vmargin=3cm]{geometry}
\usepackage{tikzpagenodes}
\usepackage{lipsum}
\usepackage{xcolor}
\usepackage{cancel}
\usepackage{amsmath}
\usepackage{mathrsfs}
\usepackage{amssymb}
\usepackage{background}
\usepackage{titlesec}
\usepackage[nodisplayskipstretch]{setspace}
\usepackage{hyphenat}
\usepackage[normalem]{ulem}
\usepackage{subcaption}
\hyphenation{mate-mática recu-perar}

\titlespacing{\section}{0pc}{-0.25em}{0pc}
\titlespacing{\subsection}{0pc}{0em}{0pc}
\titlespacing{\subsubsection}{0pc}{0.33em}{0pc}
\titlespacing{\paragraph}{0em}{0.25em}{0.5em}
\setlength{\parindent}{2em}
\setlength{\parskip}{1em}
\linespread{1}

\renewcommand{\baselinestretch}{1.0}

\renewcommand\bf[1]{\textbf{#1}}
\renewcommand\it[1]{\textit{#1}}

\newcommand\ov[1]{\overline{#1}}
\renewcommand\u[1]{\underbar{#1}}
\newcommand{\vect}[1]{\mathbf{#1}}
\newcommand{\bb}[1]{\mathbb{#1}}
\newcommand{\vn}{\varnothing}
\newcommand\stk[2][black]{\setbox0=\hbox{$#2$}%
\rlap{\raisebox{.45\ht0}{\textcolor{#1}{\rule{\wd0}{1pt}}}}#2}

\makeatletter
\global\let\tikz@ensure@dollar@catcode=\relax
\makeatother

\makeatletter
\def\mcolor#1#{\@mcolor{#1}}
\def\@mcolor#1#2#3{%
  \protect\leavevmode
  \begingroup
    \color#1{#2}#3%
  \endgroup
}
\makeatother
\definecolor{notepadrule}{RGB}{217,244,244}

\backgroundsetup{
contents={%
  \begin{tikzpicture}
    \foreach \fila in {0,...,52}
    {
      \draw [line width=1pt,color=notepadrule]
      (current page.west|-0,-\fila*12pt) -- ++(\paperwidth,0);
    }
    \draw[overlay,red!70!black,line width=1pt]
      ([xshift=-1pt]current page text area.west|-current page.north) --
      ([xshift=-1pt]current page text area.west|-current page.south);
  \end{tikzpicture}%
},
scale=1,
angle=0,
opacity=1
}

\begin{document}

\setlength{\abovedisplayskip}{12pt}
\setlength{\belowdisplayskip}{12pt}
\setlength{\abovedisplayshortskip}{0pt}
\setlength{\belowdisplayshortskip}{0pt}
% \setlength{\baselineskip}{12pt}
\setlength{\jot}{1pt}

\section{Transformadas Lineares de Vetores Aleatórios}
Considere o seguinte problema: temos um vetor $X$ com matriz de covariâncias $C_{XX}$ e queremos
definir uma função $g$ tal que $Y = g(X)$ tenha uma matriz de covariâncias diagonal. \it{Isso
significa componentes não-correlatos mas não garante independência.}

Dois aspectos:
\begin{itemize}
    \item Este problema tem solução em forma fechada;
    \item A solução é uma transformação linear;
\end{itemize}

Questiona-se o que acontece com uma matriz de covariâncias se aplicarmos uma transformada linear a
$X$. Temos um vetor aleatório $X \in S \to \mathbb{C}^N$. Temos uma matriz determinística $A$ e
definimos:
\begin{align*}
    &Y = A X \\
      &\text{Note que Y é um vetor aleatório:} \\
      &Y(s) = A X(s) \\
         &\text{Qual a matrix de covariâncias } C_{YY}?
\end{align*}
\bf{Obs:} Se $X$ é variável aleatória com variância $\sigma_X^2$. Se $Y = aX$, $\sigma_Y^2 = a^2
\cdot \sigma_X^2$

Solução:
\begin{align*}
    C_{YY} &= \mathbb{E}[(Y-\mu_Y)(Y-\mu_Y)^H] \\
    C_{YY} &= \mathbb{E}[(AX-\mu_Y)(AX-\mu_Y)^H] \\
    C_{YY} &= \mathbb{E}[(AX-\mu_Y)(AX-\mu_Y)^H] \\
    \mu_Y &= \mathbb{E}[Y] = \mathbb{E}[AX] = A\mu_X \\
    C_{YY} &= \mathbb{E}[(AX-A\mu_X)(AX-A\mu_X)^H] \\
    C_{YY} &= \mathbb{E}[(A(X-\mu_X))(A(X-\mu_X))^H] \\
    \text{lembrando que: } (AB)^T &= A^TB \text{ e que também } (AB)^H = A^HB \\
    C_{YY} &= \mathbb{E}[A(X-\mu_X)(X-\mu_X)^HA^H] \\
    C_{YY} &= A\mathbb{E}[(X-\mu_X)(X-\mu_X)^H]A^H \\
    C_{YY} &= AC_{XX}A^H \\
    \text{no caso real: } C_{YY} &= AC_{XX}A^T \\
    \text{no caso 1D: } \sigma_Y^2 &= a \cdot \sigma_X^2 \cdot a^* = |a|^2 \sigma_X^2
\end{align*}

\subsection{Desafio - Diagonalizando $C_{XX}$}
Calcule $A$ para que $C_{XX}$ seja matriz diagonal.

\section{Funções de Vetores Aleatórios}
Considere um vetor aleatório $X \in S \to \mathbb{R}^N$. Seja uma função $g: \mathbb{R}^N \to
\mathbb{R}^N$ e defina o vetor aleatório $Y \in S \to \mathbb{R}^N$ a partir de $Y(s) = g(X(s))$.

Qual a PDF de $Y$ dada a PDF de $X$?

Lembrando o caso de variáveis aleatórias: $Y = g(X)$, para cada $y \in \mathbb{R}$ consideramos
$x_i$ tais que $g(x_i) = y$ e obtivemos:
\begin{align*}
    f_Y(y) &= \sum_i \frac{f_X(x_i)}{|g'(x_i)|}
\end{align*}

No caso de vetores aleatórios:
Para um dado $y \in \mathbb{R}^N$, considerar todos os $x_i \in \mathbb{R}^N$ com $g(x_i) = y$.
\it{A função $g$ recebe um vetor de entrada e retorna um vetor de $g_N$ componentes.}

A derivada será substituida pelo Jacobiano de $g$:
\begin{align*}
    \mathrm{J}_g(f)&=\left.\begin{bmatrix}\frac{\partial f_1}{\partial x_1}&\frac{\partial f_1}{\partial x_2}&\dots&\frac{\partial f_1}{\partial x_n}\\\frac{\partial f_2}{\partial x_1}&\frac{\partial f_2}{\partial x_2}&\dots&\frac{\partial f_2}{\partial x_n}\\
\vdots&\vdots&&\vdots\\\frac{\partial f_m}{\partial x_1}&\frac{\partial f_m}{\partial
x_2}&\dots&\frac{\partial f_m}{\partial x_n}\\\end{bmatrix}\right|_x \\
\
            f_Y(y) &= \sum_i \frac{f_X(x_i)}{|det \; J_g(x_i)|}
\end{align*}

\subsection{Exercício}
Seja $X \in S \to \mathbb{C}^N$ um vetor aleatório com PDF conjunta $f_X$. Seja ainda $A$ uma
matriz $N \times N$. Determine a PDF conjunta de $Y = AX$. Considere que $A$ é não singular ($det
\; A \neq 0$).

\section{Exemplo de PDF multivariada}
PDF Gaussiana no caso monovariado:
\begin{align*}
f_X(x) &= \frac{1}{\sqrt{2\pi\sigma^2}} \exp\left({-\frac{(x-\mu)^2}{2\sigma^2}}\right)
\end{align*}
No caso multivariado:
\begin{align*}
    f_X(\u{x}) &= \frac{1}{\sqrt{(2\pi)^N det(C_{\u{x}})}} \exp\left({-\frac{1}{2} (\u{x}-\mu_{\u{x}})^T C_{XX}^{-1}(\u{x}-\mu_{\u{x}})}\right) \\
    \text{Exemplo } N &= 2, \\
    C_{XX} &=
    \begin{bmatrix}
        \sigma^1_1 & c\\
        c & \sigma^2_2
    \end{bmatrix} \qquad
    C_{XX}^{-1} = \frac{1}{\sigma_1^1 \cdot \sigma_2^2 - c^2}
    \begin{bmatrix}
        \sigma^1_2 & -c\\
        -c & \sigma^2_1
    \end{bmatrix} \\
    f_X(\u{x}) &= \frac{1}{\sqrt{(2 \pi)^2 \cdot (\sigma_1^1 \cdot \sigma_2^2 - c^2)}}
    exp\left({-\frac{1}{2} (\u{x}-\mu_{\u{x}})^T \frac{1}{\sigma_1^1 \cdot \sigma_2^2 - c^2}
        \begin{bmatrix}
            \sigma^1_2 & -c\\
            -c & \sigma^2_1
        \end{bmatrix}
    (\u{x}-\mu_{\u{x}})}\right)
\end{align*}


\end{document}
