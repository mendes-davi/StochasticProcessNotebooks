\documentclass{article}
\usepackage[T1]{fontenc}
\usepackage[utf8]{inputenc}
\usepackage[portuguese]{babel}
\usepackage[vmargin=3cm]{geometry}
\usepackage{tikzpagenodes}
\usepackage{lipsum}
\usepackage{xcolor}
\usepackage{amsmath}
\usepackage{amssymb}
\usepackage{background}
\usepackage{titlesec}
\usepackage[nodisplayskipstretch]{setspace}
\usepackage{hyphenat}
\hyphenation{mate-mática recu-perar}

\titlespacing{\section}{0pc}{-0.25em}{0pc}
\titlespacing{\subsection}{0pc}{0.25em}{0pc}
\titlespacing{\subsubsection}{0pc}{0.33em}{0pc}
\titlespacing{\paragraph}{0em}{0.125em}{0.5em}
\setlength{\parindent}{2em}
\setlength{\parskip}{1em}
\linespread{1}

\renewcommand{\baselinestretch}{1.0}

\renewcommand\bf[1]{\textbf{#1}}
\renewcommand\it[1]{\textit{#1}}

\newcommand\ov[1]{\overline{#1}}
\newcommand{\vect}[1]{\mathbf{#1}}
\newcommand{\vn}{\varnothing}

\makeatletter
\def\mcolor#1#{\@mcolor{#1}}
\def\@mcolor#1#2#3{%
  \protect\leavevmode
  \begingroup
    \color#1{#2}#3%
  \endgroup
}
\makeatother
\definecolor{notepadrule}{RGB}{217,244,244}

\backgroundsetup{
contents={%
  \begin{tikzpicture}
    \foreach \fila in {0,...,52}
    {
      \draw [line width=1pt,color=notepadrule]
      (current page.west|-0,-\fila*12pt) -- ++(\paperwidth,0);
    }
    \draw[overlay,red!70!black,line width=1pt]
      ([xshift=-1pt]current page text area.west|-current page.north) --
      ([xshift=-1pt]current page text area.west|-current page.south);
  \end{tikzpicture}%
},
scale=1,
angle=0,
opacity=1
}

\begin{document}

\setlength{\abovedisplayskip}{12pt}
\setlength{\belowdisplayskip}{12pt}
\setlength{\abovedisplayshortskip}{0pt}
\setlength{\belowdisplayshortskip}{0pt}
% \setlength{\baselineskip}{12pt}
\setlength{\jot}{0em}

\section{Interpretação de Expressões que Descrevem Probabilidades}
\paragraph{Obs:} Sempre que descrevermos textualmente uma probabilidade, devemos ser capazes de
traduzir o evento descrito como conjunto de um $S$, que tenha todos os possíveis resultados.

\bf{Exemplo:} $P(X^2 + X < 5 Y)$, interprete como um conjunto de elementos $s \in S$ tais que
$[X(s)]^2 + [X(s)] < 5 Y(s)$.

\section{Caso particular: Espaços Amostrais Finitos}
Considere $S$ um conjunto finito (de N elementos). Podemos denotar a cardinalidade de $S$ como $|S|
= N$.

Com frequência, cálculos de probabilidade de $A \subseteq S$ envolvem determinal o número de
elementos de $A$. Ademais, o problema se reduz a um problema de \bf{análise combinatorial}.

\subsection{Número binomial ${N \choose k}$}
Número de \it{subconjuntos} de k elementos tirados de um \it{conjunto} de N elementos. Refere-se ao
número de combinações de k elementos proveniente de um conjunto de N elementos. Computacionalmente,
\texttt{nchoosek(N,k)}.

\begin{align*}
    {N \choose k} = \frac{N!}{k!(N-k)!}, \text{num. de combinações} \\[0.5em]
    m!, \text{número de permutações de m elementos}
\end{align*}
As permutações são sequências usando todos os $m$ elementos uma única vez.

Esses dois valores - ${N \choose k}$ e $m!$ - são comumente usado em análises de probabilidade
referentes a conjuntos finitos em que os subconjuntos têm igual probabilidade.

\paragraph{Exemplo:}
Seja $S = \{1,2,3,4,5\}$ iid.

Nesse cenário, a probabilidade de qualquer $A \subseteq S$ é $P(A) = \frac{|A|}{|S|}$.
Lembrando que só vale para espaços amostrais finitos com todos os subconjuntos unitários
equiprováveis.

\section{Um experimento populacional...}
Conside $N_P$ indivíduos sem contato entre si, selecionados aleatoriamente de uma população.
Considere que cada indivíduo tem uma probabilidade $p$ de ter contraído COVID-19. Qual a
probabilidade de exatamente $k$ pessoas dessas $N \; (0 \leq k \leq N_P)$ tenha contraído COVID-19?

\paragraph{Definição para Eventos Independentes:}
Dado $S$ e dados $A, B \subseteq S \; (A, B \in S)$, $A$ e $B$ são independentes se e somente se:
$P(A \cap B) = P(A)P(B)$ (definição de independência)

\newpage
\subsection*{Solução:}
Usando $C$ para denotar indivíduos com COVID; $H$ indivíduos sem COVID.
\\[0.25em]
Espaço Amostral: $S = \{(C,C,\ldots,C),(C,C,\ldots,H),\ldots,(C,C,\ldots,H,H),\ldots,(H,H,\ldots,H),\ldots\}$

Cada subconjunto contém $N_P$ elementos. Temos que a cardinalidade de $S$ é: $|S| = 2^{N_P}$
\begin{align*}
    P(I^n_C) &= p \qquad I^n_C \rightarrow \text{ind. $n$ contraiu COVID-19} \\
    I^n_C &= \{(C,C_n,\ldots,C),(C,C_n,\ldots,H),\ldots,(C,C_n,\ldots,H,H),\ldots,(H,C_n,\ldots,H),\ldots\} \\\\
    P(E^{1,2,\ldots,k}_C) &= \; ? \text{ indivíduos } 1,2,3,\ldots,k \text{ contraíram COVID-19, os demais não} \\
    E^{1,2,\ldots,k}_C &= \{C_1,C_2,\ldots,C_k,H_1,H_2,\ldots,H_{N-k}\} \\
    E^{1,2,\ldots,k}_C &= I_C^1 \cap I_C^2 \cap I_C^3 \ldots \cap I_C^k \cap \ov{I_C^{k+1}} \cap \ov{I_C^{k+2}} \cap \ov{I_C^{N}} \\
    P(E^{1,2,\ldots,k}_C) &= P(I_C^1) \cdot P(I_C^2) \cdot P(I_C^k) \cdot P(\ov{I_C^{k+1}}) \cdot
    P(\ov{I_C^{k+2}}) \cdot P(\ov{I_C^N)} \\
    P(E^{1,2,\ldots,k}_C) &= \underbrace{p \cdot p \ldots \cdot p}_{k} \cdot \underbrace{(1-p)
    \cdot (1-p) \ldots (1-p)}_{N-k} \\
    P(E^{1,2,\ldots,k}_C) &= p^k \cdot (1-p)^{N-k}
\end{align*}
\\[-0.5em]
Agora, precisamos generalizar o resultado para o caso de $k$ indivíduos quaisquer, não só os $k$
primeitos. Como existem ${N \choose k}$ possibilidades de extrair $k$ pessoas a partir do grupo de
$N_P$ pessoas, e como os conjuntos $E^{\ldots}$ são disjuntos:

$P(k \text{ contamidados}) = {N \choose k} p^k (1-p)^{N-k}$, sendo $k$ contamidados a união de
todos os $E^{(\ldots)}_C$.

\end{document}
