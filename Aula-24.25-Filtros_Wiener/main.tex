\documentclass{article}
\usepackage[T1]{fontenc}
\usepackage[utf8]{inputenc}
\usepackage[portuguese]{babel}
\usepackage[vmargin=3cm]{geometry}
\usepackage{tikzpagenodes}
\usepackage{lipsum}
\usepackage{xcolor}
\usepackage{hyperref}
\usepackage{amsmath}
\usepackage{amssymb}
\usepackage{background}
\usepackage{titlesec}
\usepackage[nodisplayskipstretch]{setspace}
\usepackage{hyphenat}
\usepackage{float}
\usepackage{graphicx}
\hyphenation{mate-mática recu-perar}

\titlespacing{\section}{0em}{-0.5em}{0em}
\titlespacing{\subsection}{0pc}{0em}{0pc}
\titlespacing{\subsubsection}{0pc}{0.33em}{0pc}
\titlespacing{\paragraph}{0em}{0.5em}{0em}
\setlength{\parindent}{2em}
\setlength{\parskip}{1em}
\linespread{1}

\renewcommand{\baselinestretch}{1.0}

\renewcommand\bf[1]{\textbf{#1}}
\renewcommand\it[1]{\textit{#1}}

\newcommand\ov[1]{\overline{#1}}
\newcommand{\vect}[1]{\mathbf{#1}}
\newcommand{\vn}{\varnothing}

\DeclareMathOperator*{\argmin}{argmin} % no space, limits underneath in displays

\newenvironment{nscenter}
 {\parskip=1.25em\par\nopagebreak\centering}
 {\par\noindent\ignorespacesafterend}

\makeatletter
\def\mcolor#1#{\@mcolor{#1}}
\def\@mcolor#1#2#3{%
  \protect\leavevmode
  \begingroup
    \color#1{#2}#3%
  \endgroup
}
\makeatother
\definecolor{notepadrule}{RGB}{217,244,244}

\backgroundsetup{
contents={%
  \begin{tikzpicture}
    \foreach \fila in {0,...,52}
    {
      \draw [line width=1pt,color=notepadrule]
      (current page.west|-0,-\fila*12pt) -- ++(\paperwidth,0);
    }
    \draw[overlay,red!70!black,line width=1pt]
      ([xshift=-1pt]current page text area.west|-current page.north) --
      ([xshift=-1pt]current page text area.west|-current page.south);
  \end{tikzpicture}%
},
scale=1,
angle=0,
opacity=1
}

\begin{document}

\setlength{\abovedisplayskip}{12pt}
\setlength{\belowdisplayskip}{0.75em}
\setlength{\abovedisplayshortskip}{0pt}
\setlength{\belowdisplayshortskip}{0pt}
% \setlength{\baselineskip}{12pt}
\setlength{\jot}{1pt}

\section{Filtros de Wiener}
Há um processo estocástico $x$ contaminado por ruído $\eta$, tal que: $x_r = x + \eta$. Sendo que
$x$ é processo desejado e $x_r$ é o processo obtido.

O filtro de Wiener objetiva minimizar o valor esperado do erro quadrático entre $y$ (saída do
filtro) e $x$.
\begin{align*}
    e_n &= y_n - x_n \\
    \text{valoe esperado do erro: } &\mathbb{E}\left[\frac{1}{N} \sum_{n=0}^{N-1} |y_n - x_n|^2\right]
\end{align*}

O Filtro de Wiener é obtido minimizando esse erro médio quadrático esperado.
\begin{align*}
    h^{\star} &= \argmin_{h} \mathbb{E}\left[ \frac{1}{N} \sum_{n=0}^{N-1} | (h*x_n)_n - x_n |^2 \right] \\
    h^{\star} &= \argmin_{h} \mathbb{E}\left[ \frac{1}{N} \sum_{n=0}^{N-1}
    [h*(x+\eta)_{[n]}-x_{[n]}] [h^* * (x^* + \eta^*)_{[n]} - x_{[n]}^*] \right]
\end{align*}

O desenvolvimento das expressões envolverá os seguintes termos:
\begin{itemize}
    \item $\mathbb{E}[x_{n_1}x_{n_2}^*]$ -- $R_{XX}[n_1, n_2]$, autocorrelação
    \item $\mathbb{E}[x_{n_1}\eta_{n_2}^*]$ -- $R_{x\eta}[n_1,n_2]$ correlação cruzada
\end{itemize}

Supondo que $x$ é estacionário no sentido amplo:
\begin{align*}
    r_x[n_2-n_1] &= R_{XX}[n_1,n_2]
\end{align*}

\bf{Supondo que $x$ e $\eta$ são não-correlatos e que a média do ruído é nula e WSS também}:
\begin{align*}
    R_{X\eta}[n_1,n_2] &= 0 \\
    r_{\eta}[n_2-n_1] &= R_{\eta\eta}[n_1,n_2]\\[-2.5em]
\end{align*}
Pontos de Interesse:
\begin{itemize}
    \setlength\itemsep{0em}
    \item Sinal e Ruído não correlatos para zerar a covariância cruzada entre sinal e ruído
    \item expectância do ruído é nula, para todos os instantes do tempo (lembrando que $\eta$ é
        suposto WSS)
\end{itemize}

Solução:
\begin{align*}
    H(f) &= \frac{PSD_x(f)}{PSD_x(f)+PSD_\eta(f)}, \forall f \in \mathbb{R}
\end{align*}
Supondo que $\eta$ é branco (WSS e PSD cte.): $PSD_\eta(f) = k$, obtemos:
\begin{align*}
    H(f) &= \frac{PSD_x(f)}{PSD_x(f)+k}
\end{align*}

\section{Processos de Markov}
Considere um processo $X: \mathbb{S} \to \mathcal{E}_S$. Considere instantes de tempo (ordenados) dados por
$t_1, t_2, t_3, \ldots, t_N$ que fazem parte do domínio de cada sinal em $\mathcal{E}_S$.

O processo $X$ é um processo de Markov se e somente se:
\begin{align*}
    f_X(t_N,x_N|t_{N-1},x_{N-1};t_{N-2},x_{N-2};\ldots;t_1,x_1) = f_X(t_N,x_N|t_{N-1},x_{N-1})\\[-1.75em]
\end{align*}

Em particular, as cadeias de Markov são processos de Markov correspondentes a sinais em tempo
discreto em cada sinal pode assumir uma quantidade finita de valores, que passam as ser denominados
\textit{estados}.



\end{document}
