\documentclass{article}
\usepackage{enumitem}
\usepackage[T1]{fontenc}
\usepackage[utf8]{inputenc}
\usepackage[portuguese]{babel}
\usepackage[vmargin=3cm]{geometry}
\usepackage{tikzpagenodes}
\usepackage{lipsum}
\usepackage{xcolor}
\usepackage{cancel}
\usepackage{amsmath}
\usepackage{amssymb}
\usepackage{background}
\usepackage{titlesec}
\usepackage[nodisplayskipstretch]{setspace}
\usepackage{hyphenat}
\usepackage[normalem]{ulem}
% \usepackage{minted}
\usepackage{subcaption}
\hyphenation{mate-mática recu-perar}

\titlespacing{\section}{0pc}{-0.25em}{0pc}
\titlespacing{\subsection}{0pc}{0em}{0pc}
\titlespacing{\subsubsection}{0pc}{0.33em}{0pc}
\titlespacing{\paragraph}{0em}{0.125em}{0.5em}
\setlength{\parindent}{2em}
\setlength{\parskip}{1em}
\linespread{1}

\renewcommand{\baselinestretch}{1.0}

\renewcommand\bf[1]{\textbf{#1}}
\renewcommand\it[1]{\textit{#1}}

\newcommand\ov[1]{\overline{#1}}
\newcommand{\vect}[1]{\mathbf{#1}}
\newcommand{\vn}{\varnothing}
\newcommand\stk[2][black]{\setbox0=\hbox{$#2$}%
\rlap{\raisebox{.45\ht0}{\textcolor{#1}{\rule{\wd0}{1pt}}}}#2}

\makeatletter
\global\let\tikz@ensure@dollar@catcode=\relax
\makeatother

\makeatletter
\def\mcolor#1#{\@mcolor{#1}}
\def\@mcolor#1#2#3{%
  \protect\leavevmode
  \begingroup
    \color#1{#2}#3%
  \endgroup
}
\makeatother
\definecolor{notepadrule}{RGB}{217,244,244}

\backgroundsetup{
contents={%
  \begin{tikzpicture}
    \foreach \fila in {0,...,52}
    {
      \draw [line width=1pt,color=notepadrule]
      (current page.west|-0,-\fila*12pt) -- ++(\paperwidth,0);
    }
    \draw[overlay,red!70!black,line width=1pt]
      ([xshift=-1pt]current page text area.west|-current page.north) --
      ([xshift=-1pt]current page text area.west|-current page.south);
  \end{tikzpicture}%
},
scale=1,
angle=0,
opacity=1
}

\begin{document}

\setlength{\abovedisplayskip}{12pt}
\setlength{\belowdisplayskip}{12pt}
\setlength{\abovedisplayshortskip}{0pt}
\setlength{\belowdisplayshortskip}{0pt}
% \setlength{\baselineskip}{12pt}
\setlength{\jot}{0pt}

\section{Probabilidade Condicional}
Seja $S$ um espaço amostral e seja $\mathbb{E}$ um conjunto de eventos - $\mathbb{E} \subseteq
P(S), \vn \in \mathbb{E}, S \in \mathbb{E}$. Seja $P:\mathbb{E} \rightarrow [0,1]$ uma função de
probabilidade. Neste caso, se $A \in \mathbb{E}$ tal que $P(A) > 0$, então a probabilidade de $B
\in \mathbb{E}$ condicionada por a $A$ é definida por:
\begin{align*}
    P(B|A) = \frac{P(A \cap B)}{P(A)}
\end{align*}
\\[0.5em]
\bf{OBS:} Na probabilidade condicional, reescalonamos as probabilidades (dividindo por $P(A)$) de
forma que $P(A|A)$ seja 1. \bf{É como se $A$ fosse o novo espaço amostral.}
\\
$f(B) = P(B|A)$: esta função satisfaz os axiomas de Kolmogorov se considerarmos um novo $S = A$ e
se substituirmos $B$ por $B \cap A$. \bf{Isso significa que a probabilidade condicional} é também
uma função de probabilidade, porém com o espaço amostral mais restrito que o original.

\subsection*{Exemplo:}
Jogamos um dado equilibrado de $6$ faces e sabendo que o resultado é ímpar e maior que 3.
\begin{align*}
    S &= \{1,2,3,4,5,6\} \\
    A &= \{4,5,6\} \\
    B &= \{5\} \\
    P(B|A) &= \frac{P(A \cap B)}{P(A)} = \frac{P(\{5\})}{P(\{4,5,6\})} = \frac{1/6}{1/2} =
    \frac{1}{3}
\end{align*}

\section{Análise Bayesiana (de \it{Bayes})}
\bf{Ex:} Jogador de basquete com alta probabilidade de acerto em lance livre: $P(A|J_1) = 0.9$.
Jogador amador com probabilidade de acerto baixa: $P(A|J_2) = 0.1$. Um dos jogadores é sorteado,
com probabilidades diferentes: $P(J_1) = 0.05$ e $P(J_2) = 0.95$.
\\
\it{Pergunta}: se um jogador foi sorteado e acertou, qual a probabilidade de que tenha sido
$J_1$? \it{[Probabilidade a posteriori]}
\begin{align*}
    P(J_1|A) = \frac{P(J_1 \cap A)}{P(A)} =? && \\
    P(A|J_1) = \frac{P(A \cap J_1)}{P(J_1)} && P(A|J_2) = \frac{P(A \cap J_2)}{P(J_2)} \\
    P(A \cap J_1) = P(A|J_1)P(J_1) &&  P(A \cap J_2) = P(A|J_2)P(J_2) \\
    \\
    P(A) =  P((A \cap J_1) \cup (A \cap J_2)) && \text{pois } (A \cap J_1) \cup (A \cap J_2) = A \\
    \\
    \text{Usando o axioma 3:} && \\
        && P(A) = P(A \cap J_1) P(A \cap J_2) \\
        && P(A) = P(A|J_1)P(J_1) + P(A|J_2)P(J_2) \\
    \\
    \text{Teorema de Bayes:}&&{P(J_1|A) = \frac{P(A|J_1)P(J_1)}{P(A)}}\\
\end{align*}

Podemos realizar a substituição para obter:
\begin{align*}
    \underbrace{P(J_1|A) = \frac{P(A|J_1)P(J_1)}{P(A|J_1)P(J_1) + P(A|J_2)P(J_2)}}\\{\text{Caso
    em que $J_1 \cap J_2 = \vn$ e $J_1 \cup J_2 = S$}} \\
    \text{($J_1$ e $J_2$ são partição de $S$)}
\end{align*}
\\[-0.5em]
Isso leva a generalização do Teorema de Bayes:
\begin{align*}
    P(B|A) = \frac{P(A|B)P(B)}{P(A|C_1)P(C_1) +P(A|C_2)P(C_2) + \ldots +P(A|C_N)P(C_N)}
\end{align*}

\bf{Resultado para o exemplo:}
\begin{align*}
    P(J_1|A) &= \frac{0.9 \cdot 0.05}{0.9 \cdot 0.05 + 0.1 \cdot 0.95} = 0.32143 \\
    P(J_2|A) &= 0.67857
\end{align*}
\\[-2em]
\section{Distribuições Condicionais}
Seja uma variável $X: S \rightarrow \mathbb{R}$. Seja ainda $A \in \mathbb{P}(S)$ para o qual é
definida a probabilidade $P(A)$, com $A_X$ a imagem de $A$ por $X$: $A_X = \{x \in \mathbb{R} /
    \exists \; s \in A \text{ com } X(s) = x\}$

Definição PDF Condicionada:

$f_{X|A_X}(x|A_x) = \frac{f_X(x)}{P_X(A_X)} = \frac{f_X(x)}{P(A)}$, se $x \in A_X$.

$f_{X|A_X}(x|A_x) = 0$, caso contário

\bf{Obs:} $\frac{f_X(x)}{P_X(A_X)} = \frac{f_X(x)}{\int_{A_X} f_X(x)dx}$

Teorema: A integral da PDF condicionada revela a probabilidade condicional.
\begin{align*}
    \int_{B_X} f_{X|A_X}(x|A_X)dx = P_X(B_X|A_X)
\end{align*}

\end{document}
