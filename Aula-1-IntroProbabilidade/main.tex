\documentclass{article}
\usepackage[T1]{fontenc}
\usepackage[utf8]{inputenc}
\usepackage[portuguese]{babel}
\usepackage[vmargin=3cm]{geometry}
\usepackage{tikzpagenodes}
\usepackage{lipsum}
\usepackage{xcolor}
\usepackage{amsmath}
\usepackage{amssymb}
\usepackage{background}
\usepackage{titlesec}
\usepackage[nodisplayskipstretch]{setspace}
\usepackage{hyphenat}
\hyphenation{mate-mática recu-perar}

\titlespacing{\section}{0pc}{-0.25em}{0pc}
\titlespacing{\subsection}{0pc}{0em}{0pc}
\titlespacing{\subsubsection}{0pc}{0.33em}{0pc}
\titlespacing{\paragraph}{0em}{0.125em}{0.5em}
\setlength{\parindent}{2em}
\setlength{\parskip}{1em}
\linespread{1}

\renewcommand{\baselinestretch}{1.0}

\renewcommand\bf[1]{\textbf{#1}}
\renewcommand\it[1]{\textit{#1}}

\newcommand\ov[1]{\overline{#1}}
\newcommand{\vect}[1]{\mathbf{#1}}
\newcommand{\vn}{\varnothing}

\makeatletter
\def\mcolor#1#{\@mcolor{#1}}
\def\@mcolor#1#2#3{%
	\protect\leavevmode
	\begingroup
	\color#1{#2}#3%
	\endgroup
}
\makeatother
\definecolor{notepadrule}{RGB}{217,244,244}

\backgroundsetup{
	contents={%
			\begin{tikzpicture}
				\foreach \fila in {0,...,52}
					{
						\draw [line width=1pt,color=notepadrule]
						(current page.west|-0,-\fila*12pt) -- ++(\paperwidth,0);
					}
				\draw[overlay,red!70!black,line width=1pt]
				([xshift=-1pt]current page text area.west|-current page.north) --
				([xshift=-1pt]current page text area.west|-current page.south);
			\end{tikzpicture}%
		},
	scale=1,
	angle=0,
	opacity=1
}

\begin{document}

\setlength{\abovedisplayskip}{12pt}
\setlength{\belowdisplayskip}{12pt}
\setlength{\abovedisplayshortskip}{0pt}
\setlength{\belowdisplayshortskip}{0pt}
% \setlength{\baselineskip}{12pt}
\setlength{\jot}{0pt}

\section{Aplicando a Teoria dos Conjuntos em Probabilidade}
A probabilidade é baseada em um experimento repetitivo que consiste em um procedimento e
observações. Um \it{resultado} é uma observação. Um \it{evento} é um conjunto de resultados.

\paragraph{Resultado:}
Um \bf{resultado} de um \it{experimento} é qualquer observação possível desse experimento. Segundo
a noção de que os experimentos são distinguíveis entre si definimos um conjunto universal de todos
os resultados possíveis.

\paragraph{Espaço Amostral $S$:}
O \bf{espaço amostral} $S$ de um \it{experimento} é o conjunto mais minucioso, mutuamente
exclusivo, coletivamente completo de todos os resultados possíveis.

\paragraph{Evento:}
Um \bf{evento} é um conjunto dos resultados de um \it{experimento}.

\paragraph{Conjunto das partes de $S$:}
$\mathbb{P}(S)$ conjunto de todos os subconjuntos de $S$. Ex: $S = \{1,2,3\}$

\begin{itemize}
	\setlength\itemsep{-0.5em}
	\item $\{1,2\}$ é um subconjunto de $S$
	\item $\{1\}$ é um subconjunto de $S$
	\item $\{3\}$ é um subconjunto de $S$
	\item $\{\vn\}$ é um subconjunto de $S$
	\item \ldots
\end{itemize}

\paragraph{Obs:} Note $\vn = \{\cdot\}$ \bf{entretanto} $\{\vn\} \neq \vn$

\section{Axiomas de Probabilidade}
Uma medida de probabilidade $P[\cdot]$ é uma função que mapeia eventos no espaço amostral a
números reais tais que:


\begin{enumerate}
	\setlength\itemsep{0em}
	\item Para qualquer evento $\mathbb{E}$, $P[\mathbb{E}] \geq 0$
	\item $P[S] = 1$
	\item Para qualquer coleção contável $A_1, A_2,\ldots$ de eventos mutuamente exclusivos:\\ $P[A_1
				      \cup A_2 \cup \ldots] = P[A_1] + P[A_2] + \ldots$
\end{enumerate}

Toda a teoria de probabilidade é baseada nesses três axiomas. Os axiomas 1 e 2 estabelecem a
probabilidade como um número entre 0 e 1. O axioma 3 mostra que \bf{para conjuntos disjuntos} a
união de conjuntos corresponde a soma das probabilidades.

\subsection{Teoremas e Consequências dos Axiomas}
\subsubsection{$P(\vn) = 0$}
Usando o axioma 3 note que:
\begin{align*}
	P(\vn \cup \vn \cup \vn \cup \ldots) = \sum_{i=1}^{\infty} P(\vn) \text{, pois } \vn \cap \vn = \vn \\
	P(\vn) = \sum_{i=1}^{\infty} P(\vn) \text{ só converge se } P(\vn) = 0
\end{align*}

\subsubsection{$P(A \cap \ov{B}) = P(A) - P(A \cap B) \; \forall \; A,B \in \mathbb{E}$}
Manuseando a equação para mostrar que $P(A \cap \ov{B}) + P(A \cap B)= P(A)$. Primeiramente:
\begin{align*}
	(A \cap \ov{B}) \cap (A \cap B)                & = \vn \text{, da álgebra Booleana: } A(\ov{B} + B) = \mcolor{blue}{(A\ov{B} + AB)} \\
	\mcolor{blue}{(A \cap \ov{B}) \cup (A \cap B)} & = A \cap (\ov{B} \cup B) = A \cap S = A                                            \\
	P(A \cap \ov{B})                               & = P(A) - P(A \cap B)
\end{align*}

\subsubsection{$P(\ov{A}) = 1 - P(A)$}
Utilizando os Axiomas 3 e 2 fazemos:
\begin{align*}
	S         & = A \cup \ov{A}    \\
	1         & = P(A) + P(\ov{A}) \\
	P(\ov{A}) & = 1 - P(A)
\end{align*}

\subsubsection{$P(A \cup B) = P(A) + P(B) - P(A \cap B) \; \forall \; A,B \in \mathbb{E}$}
Notamos que $A \cap \ov{B}$ e $B$ são disjuntos e tem união $A \cup B$. Então:
\begin{align}
	P(A \cup B) = P(B) + P(A \cap \ov{B}) \label{eq:t4_1}
\end{align}
Em seguida, notamos que $A \cap \ov{B}$ e $A \cap B$ são disjuntos e tem união $A$. Então:
\begin{align}
	P(A) = P(A \cap B) + P(A \cap \ov{B}) \label{eq:t4_2}
\end{align}
Subtraindo \ref{eq:t4_2} de \ref{eq:t4_1}, obtemos:
\begin{align*}
	P(A \cup B) = P(A) + P(B) - P(A \cap B)
\end{align*}

\subsubsection{$0 \leq P(A) \leq 1, \; \forall \; A \in \mathbb{E}$}
Como qualquer evento $A \subseteq S$, $P(A) \leq P(S)$ então $P(A) \leq 1$. O limite inferior deve
ser zero fazendo com que: $0 \leq P(A) \leq 1$.


\end{document}
