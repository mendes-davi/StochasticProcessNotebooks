\documentclass{article}
\usepackage[T1]{fontenc}
\usepackage[utf8]{inputenc}
\usepackage[portuguese]{babel}
\usepackage[vmargin=3cm]{geometry}
\usepackage{tikzpagenodes}
\usepackage{lipsum}
\usepackage{xcolor}
\usepackage{hyperref}
\usepackage{amsmath}
\usepackage{amssymb}
\usepackage{background}
\usepackage{titlesec}
\usepackage[nodisplayskipstretch]{setspace}
\usepackage{hyphenat}
\hyphenation{mate-mática recu-perar}

\titlespacing{\section}{0em}{-0.5em}{0em}
\titlespacing{\subsection}{0pc}{0em}{0pc}
\titlespacing{\subsubsection}{0pc}{0.33em}{0pc}
\titlespacing{\paragraph}{0em}{0.5em}{0em}
\setlength{\parindent}{2em}
\setlength{\parskip}{1em}
\linespread{1}

\renewcommand{\baselinestretch}{1.0}

\renewcommand\bf[1]{\textbf{#1}}
\renewcommand\it[1]{\textit{#1}}

\newcommand\ov[1]{\overline{#1}}
\newcommand{\vect}[1]{\mathbf{#1}}
\newcommand{\vn}{\varnothing}

\newenvironment{nscenter}
 {\parskip=1.25em\par\nopagebreak\centering}
 {\par\noindent\ignorespacesafterend}

\makeatletter
\def\mcolor#1#{\@mcolor{#1}}
\def\@mcolor#1#2#3{%
  \protect\leavevmode
  \begingroup
    \color#1{#2}#3%
  \endgroup
}
\makeatother
\definecolor{notepadrule}{RGB}{217,244,244}

\backgroundsetup{
contents={%
  \begin{tikzpicture}
    \foreach \fila in {0,...,52}
    {
      \draw [line width=1pt,color=notepadrule]
      (current page.west|-0,-\fila*12pt) -- ++(\paperwidth,0);
    }
    \draw[overlay,red!70!black,line width=1pt]
      ([xshift=-1pt]current page text area.west|-current page.north) --
      ([xshift=-1pt]current page text area.west|-current page.south);
  \end{tikzpicture}%
},
scale=1,
angle=0,
opacity=1
}

\begin{document}

\setlength{\abovedisplayskip}{12pt}
\setlength{\belowdisplayskip}{12pt}
\setlength{\abovedisplayshortskip}{0pt}
\setlength{\belowdisplayshortskip}{0pt}
% \setlength{\baselineskip}{12pt}
\setlength{\jot}{0pt}

\section*{Lista 1 - Processos Estocásticos}
Davi de Alencar Mendes (16/0026415) \url{dmendes@aluno.unb.br}
\\
\section*{Problema 1}
\begin{nscenter}
Apresente a definição formal de probabilidade, enunciando os axiomas de Kolmogorov. Especifique
claramente o tipo de conjunto que constitui o domínio da função probabilidade $P[\cdot]$, em termos
de um espaço amostral $S$.
\end{nscenter}
Uma medida de probabilidade $P[\cdot]$ é uma função que mapeia eventos no espaço amostral a
números reais tais que:

\begin{enumerate}
    \setlength\itemsep{0em}
    \item Para qualquer evento $\mathbb{E}$, $P[\mathbb{E}] \geq 0$
    \item $P[S] = 1$
    \item Para qualquer coleção contável $A_1, A_2,\ldots$ de eventos mutuamente exclusivos:\\ $P[A_1
        \cup A_2 \cup \ldots] = P[A_1] + P[A_2] + \ldots$
\end{enumerate}

\section*{Problema 2}
\subsubsection*{a) $P(A \cup B) = P(A) + P(B) - P(A \cap B) \; \forall \; A,B \in \mathbb{E}$}
Notamos que $A \cap \ov{B}$ e $B$ são disjuntos e tem união $A \cup B$. Então:
\begin{align}
    P(A \cup B) = P(B) + P(A \cap \ov{B}) \label{eq:t4_1}
\end{align}
Em seguida, notamos que $A \cap \ov{B}$ e $A \cap B$ são disjuntos e tem união $A$. Então:
\begin{align}
    P(A) = P(A \cap B) + P(A \cap \ov{B}) \label{eq:t4_2}
\end{align}
Subtraindo \ref{eq:t4_2} de \ref{eq:t4_1}, obtemos:
\begin{align*}
    P(A \cup B) = P(A) + P(B) - P(A \cap B)
\end{align*}

\subsubsection*{b) $P(\ov{A}) = 1 - P(A)$}
Utilizando os Axiomas 3 e 2 fazemos:
\begin{align*}
    S &= A \cup \ov{A} \\
    1 &= P(A) + P(\ov{A}) \\
    P(\ov{A}) &= 1 - P(A)
\end{align*}

\section*{Problema 3}
\begin{nscenter}
Seja $S$ um espaço amostral, e seja $A_i$ um subconjunto de $S$ pertencente ao domínio da função de probabilidade $P, \; \forall \; \in \{1,2,3,\ldots,N\}$ com $N$ um valor inteiro positivo.
\\
Demonstre que $P(\bigcup^N_{i=1} A_i) \leq \sum^N_{i=1} P(A_i)$.
\end{nscenter}
Sabemos que a união de dois eventos é dada por: $P\{A\} = P\{A_1\} + P\{A_2\} - P\{A_1 A_2\}$.

De maneira similar, a união de três eventos é dada por: $P\{A\} = P\{A_1\} + P\{A_2\} + P\{A_3\}
- P\{A_1 A_2\} - P\{A_1 A_3\} - P\{A_2 A_3\}$

Tomando $p_i = P\{A_i\}, p_{i,j} = P\{A_i A_j\}, p_{i,j,k} = P\{A_i A_j A_k\} \ldots$

Nota-se que a união de 2 eventos é: $P\{A\} = \sum p_i - \sum p_{i,j}$, 3 eventos: $P\{A\} = \sum p_i - \sum p_{i,j} + \sum p_{i,j,k}$

Para a soma de cada probabilidade $p$, fazemos: $S_1 = \sum p_i, S_2 = \sum p_{i,j}, S_3 = \sum
p_{i,j,k}, \ldots$ com $i < j < k \leq N$, de maneira que a soma de cada combinação ocorra uma
única vez e $S_k$ contenha ${N \choose k}$ termos e $S_N$ (último termo) é a realização simultânea
de todos os $N$ eventos.

Finalmente, para a união de N eventos: $P(\bigcup_{i=1}^N A_i) = S_1 - S_2 + S_3 \ldots \pm
S_N$.

Se a união dos eventos for mutuamente exclusiva fica valendo o axioma 3 e a união dos $N$
eventos é a soma das probabilidades (ou seja $S_i = 0 \; \forall \; i > 1$).

Não sendo mutuamente exclusivos, $P(\bigcup_{i=1}^N A_i) < \sum_{i=1}^N P(A_i)$.

\section*{Problema 4}
\begin{nscenter}
Qual a noção intuitiva que se busca incorporar na definição formal de probabilidade obtida com os
axiomas de Kolmogorov?
\end{nscenter}
A probabilidade deve ser uma medida que mapeia um conjunto de eventos a um número real entre 0 e
1.

\section*{Problema 5}
\paragraph{a)} Qual a noção intuitiva que se busca capturar com os conceitos de variáveis
aleatórias e de processos estocásticos?

As variáveis aleatórias mapeiam eventos em um conjunto numéricos e provêm uma abstração
simplificada para denotar probabilidade de conjuntos. Processos estocásticos extendem essa ideia
para sequências aleatórias e complementam as ferramentas matemáticas de variáveis aleatórias.

\paragraph{b)} Por que um processo aleatório não é simplesmente uma coletânea de variáveis
aleatórias?

A abstração de processos aleatório como uma coletânea de variáveis não nos permite estudar as
propriedades inerentes a ideia de um processo como um todo e limitaria o escopo do estudo.

\paragraph{c)} Por que o conceito de probabilidade é essencial para o entendimento de processos
estocásticos?

Toda a teoria de probabilidade é fundamentada na definição axiomática da probabilidade.

\section*{Problema 6}
\begin{nscenter}
Uma urna contém três bolas numeradas de 1 a 3. Um experimento consiste em remover aleatoriamente
uma bola, registrar o número retirado, e recolocar a bola na urna antes que uma próxima
retirada(amostragem com reposição). Calcule a probabilidade de que seja obtida a mesma bola em
duas retiradas.
\end{nscenter}
O espaço amostral contém $3^2$ resultados. A combinação de 2 resultados iguais ocorre em $3$
casos. A probabilidade de retirar 2 bolas iguais é $1/9$.

\section*{Problema 7}
\begin{nscenter}
Uma urna contém três bolas numeradas de 1 a 5. Um experimento consiste em remover aleatoriamente
uma bola, registrar o número retirado, e recolocar a bola na urna antes que uma próxima
retirada(amostragem com reposição). Calcule a probabilidade de que a soma dos números obtidos nas
duas bolas seja 5.
\end{nscenter}
O espaço amostral contêm $5^2$ resultados. A soma é 5 quando são retiradas as bolas 4 e 1 ou
vice-versa, ou quando são retiradas as bolas 3 e 2 ou vice-versa. Em 4 dos 25 casos a soma é 5 com
uma probabilidade de $4/25$.

\section*{Problema 8}
\begin{nscenter}
Um experimento consiste em retirar aleatoriamente duas bolas numeradas de 1 a 5, sem reposição.
Calcule a probabilidade de que a soma seja 5.
\end{nscenter}
O espaço amostral consiste em ${5 \choose 2} \cdot 2! = 20$ amostras. Novamente, existem 4 casos
em que a soma é 5 (mesmo sem reposição os casos previamente citados são permitidos).  A
probabilidade de que a soma seja 5 é $4/20 = 1/5$

\it{That's all folks}

\end{document}
